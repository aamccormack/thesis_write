% DISCUSSION  related to the reproducibility of this technique
\chapter{Discussion of Framework for Determing Resilience}

\section{Demographics}
The purpose of this research is to assist in the creation of a framework to determine resilience of a place, specifically to sea level extremes. This is to allow repeated measurements of resilience to determine whether a place is making improvements in its resilience. How resilience is improved is done through many techniques including......

Resilience and vulnerability are impacted by demographic variables \cite{rod_integrated_2012}, but improving of resilience shouldn't be done by changing population demographics. Arguing a place is resilient simply due to their being a high percentage of women, or immigrants, or the age of the population is not a use full nor moral lens in which to discuss improvements to resilience. It can be useful to highlight areas or groups who are particularly vulnerable, but not forcing those groups to disperse against their will. For this reason standard demographic questions were not part of this survey. Subjects attributes which local governance could reasonably, cost effectively and most importantly morally impact were prioritised as research variables over demographics. 

Furthermore by asking questions upon gender and age can influence how subjects answer these questions.*citation needed*

Awareness or a hazard is not determined by gender, even if gender can correlate with it.  Gender was determined not a key variable from literature review before the creation of the survey. This is the same for the majority of demographic variables which were not investigated during this research.

Certain demographic questions can be answered by inferring from the results to key questions. For example language skills and even immigration status can be inferred from the subjects decision to complete the survey in Norwegian or English. The lack of availability of the survey in other languages is a limiting factor. Percentage of the population who do not have a reasonable understanding of either Norwegian or English in Trondheim is ****. For this reason plus limitations of funds, researcher skills and time it was deemed acceptable for the posters, emails, social media, website and surveys to only be available in Norwegian and English. However, if this was to be repeated in other locations or nationally this would need to be reconsidered. 

The key demographic that was considered was interest level and professional experience with sea level extremes. ***** highlights that those with professional experience of the relevant hazard or have professional interest are more aware and hence improve an places resilience to the specific hazard. Unfortunately even with an extended deadline marine workers were not an included part of this research. There is a small percentage who have professional interest in sea level extremes see table** below. 

\begin{table}[!ht]
    \centering
    \caption{Interest Level of Subjects in Sea Level Extremes}
    \begin{tabular}{|l|l|l|l|l|}
    \hline
        Not & Low & Medium & High & Professional \\ \hline
        1 & 2 & 3 & 4 & 5 \\ \hline
        8 & 22 & 71 & 40 & 12 \\ \hline
    \end{tabular}
\end{table}

As can be seen above the majority have medium level interest in sea level extremes. The spread of results to this question is approximatley normal, which could be seen as either that a good spread of subjects were contacted. However whether this does correctly model Trondheim's population level of interest in the subject is not very likely.  Survey participants were self selecting so that subjects have volunteered their time even with low or no interest in the research is helpful, but it is to be expected that people with an interest in a subject are more likely to take part in a survey.

This has been attempted to be minimised by making the survey as short and as easy to do as possible and by utilizing the researchers network. For more discussion on the careful utilisation of the researchers network to reach subjects who could have been missed find it under the section Survey Access. 

\section{Survey Access}

\begin{table}[!ht]
    \centering
    \caption{Access to Survey}
    \begin{tabular}{|l|l|l|l|l|l|}
    \hline
      poster & email & social media & via organisation membership & via place of employment & personal connection to researcher \\ \hline
        1 & 2 & 3 & 4 & 5 & 6 \\ \hline
        70 & 2 & 48 & 2 & 4 & 24 \\ \hline
    \end{tabular}
\end{table}

\section{Summary of Discussion of Framework - Lessons Learned}