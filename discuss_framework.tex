% DISCUSSION  related to the reproducibility of this technique
\chapter{Discussion of Framework for Determing Resilience}

\section{Software Consideration}

The ease of providing data was a key design factor considered when designing the data collection method for this project. The goal was to create an online survey which could be done by the majority of participants in under 5 minutes, without missing out on potentially valuable data. This required a balancing act of maximising potential data gathered from a single participant against maximising number of participants. To do this several iterations were created and trialled.
\paragraph{}
Due to ease of later analysis, ease of repeatability, accessibility,  avoidance of disease spread and other major benefits an online survey method was chosen. The major downside of this technique is that it could exclude certain sectors of society due to lack of access to the internet. However, in Norway over 99 percent of the population have access to the internet (ssb https://www.ssb.no/en/statbank/table/11000/tableViewLayout1/ ) *MAYBE GO INTO MORE DETAIL HERE, E.G. SMARTPHONE ACCESS*
\paragraph{}
Once online surveys were the chosen method then which software was most appropriate needed to be considered. To do so many softwares were explored and then basic surveys were created and then tested on desktops, laptops, tablets, IOS smartphones and Android Smartphones on the perceived best candidates. The results of this exploration are outlined on table** below.

\begin{table}[!ht]
    \centering
    \begin{tabular}{|l|l|l|l|l|}
    \hline
        software & nettskjema & Google forms & Arcgis survey 123 & Survey Monkey \\ \hline
        ~ & ~ & ~ & ~ & ~ \\ \hline
        Degree of security & high & high & medium & Medium \\ \hline
        Degree of privacy & high & medium & medium & Medium \\ \hline
        Link strength & High & medium & low & low \\ \hline
        Range of question formats & medium & medium & low & high \\ \hline
        Ease of use & medium & high & medium & high \\ \hline
        Gathers GIS data & no & no & yes & no \\ \hline
        Can upload pictures & no & yes & no & yes \\ \hline
        Can include several images in survey & yes & yes & no & yes \\ \hline
        Number of Surveys allowed & Unlimited & unlimited & Unlimited & limited \\ \hline
        Number of questions allowed & unlimited & unlimited & Unlimited – but over 10 questions proved unreliable & Limited \\ \hline
        Download responses as Excel or CSV & yes & yes & yes & yes \\ \hline
        downsides & New to researcher, Norway Universities specific & Requires google account for participants to log in & Slow to load on mobiles & Limited services on free subscription \\ \hline
    \end{tabular}
    \caption{Software Considerations}
    \label{table: software-considerations}
\end{table}

After this ArcGIS survey 123 was the chosen method of collection. The plan being the participants could draw upon an uploaded map where they believed the high-tide, current storm surge, past storm surge and future storm surges would be. This was believed to be the best technique as it was felt it would influence the participants answers least and allow for interesting GIS analysis later. 
\paragraph{}
However, difficulties arouse due to ArcGIS survey 123 struggling with more than one map being used in a single survey and difficulties of inputting this information on a smartphone. For this reason, a pilot survey was created in both ArcGIS survey 123 and Nettskjema. While ArcGIS survey 123 allowed for input of GIS data, Nettskjema does not. So, to compare survey methods maps of one of the selected areas were to allow for similar data input as can be seen in the sections below. 

\section{Lessons Learned from Pilot Survey}
14 participants completed the pilot survey. These participants were selected due to their high level of knowledge about changing sea levels with the majority being Natural Resource Management Students (ooh and 1 postdoc) at NTNU and the rest being members of Trondheim Kayak Klubb with an interest in sea levels and climate change. Nine out of 14 respondents selected level of concern about climate change as maximum with the rest picking the next highest level. All but two participants said they got their information from education or personal research with two respondents saying they got it from the researcher. 
\paragraph{}
%graph or table here
\begin{table}[!ht]
    \centering
    \begin{tabular}{|l|l|l|l|}
    \hline
        Question & Which image displays Brattøra's  & Which image displays Brattøra's & Which image displays   \\ \newline
         & 20-year storm surge now? &  20-year storm surge in 2090? & Brattøra's high tide? \\ \hline
        a & 3 & 1 & 4 \\ \hline
        b & 7 & 3 & 2 \\ \hline
        c & 3 & 6 & 4 \\ \hline
        d & 1 & 4 & 4 \\ \hline
        Correct Answer & B & B & C \\ \hline
    \end{tabular}
    \caption{Pilot Survey Results}
    \label{table: pilot-survey}
\end{table}

As could be inferred there appears to be no major agreement in results for the pilot survey participants. This could be due to lack of awareness, unlike what was expected but to check a focus group was run. 

\section{Lessons Learned from Focus Group}

\section{Demographics}
The purpose of this research is to assist in the creation of a framework to determine resilience of a place, specifically to sea level extremes. This is to allow repeated measurements of resilience to determine whether a place is making improvements in its resilience. How resilience is improved is done through many techniques including......

Resilience and vulnerability are impacted by demographic variables \cite{rod_integrated_2012}, but improving of resilience shouldn't be done by changing population demographics. Arguing a place is resilient simply due to their being a high percentage of women, or immigrants, or the age of the population is not a use full nor moral lens in which to discuss improvements to resilience. It can be useful to highlight areas or groups who are particularly vulnerable, but not forcing those groups to disperse against their will. For this reason standard demographic questions were not part of this survey. Subjects attributes which local governance could reasonably, cost effectively and most importantly morally impact were prioritised as research variables over demographics. 

Furthermore by asking questions upon gender and age can influence how subjects answer these questions.*citation needed*

Awareness or a hazard is not determined by gender, even if gender can correlate with it.  Gender was determined not a key variable from literature review before the creation of the survey. This is the same for the majority of demographic variables which were not investigated during this research.

Certain demographic questions can be answered by inferring from the results to key questions. For example language skills and even immigration status can be inferred from the subjects decision to complete the survey in Norwegian or English. The lack of availability of the survey in other languages is a limiting factor. Percentage of the population who do not have a reasonable understanding of either Norwegian or English in Trondheim is ****. For this reason plus limitations of funds, researcher skills and time it was deemed acceptable for the posters, emails, social media, website and surveys to only be available in Norwegian and English. However, if this was to be repeated in other locations or nationally this would need to be reconsidered. 

The key demographic that was considered was interest level and professional experience with sea level extremes. ***** highlights that those with professional experience of the relevant hazard or have professional interest are more aware and hence improve an places resilience to the specific hazard. Unfortunately even with an extended deadline marine workers were not an included part of this research. There is a small percentage who have professional interest in sea level extremes see table** below. 

\begin{table}[!ht]
    \centering
    \begin{tabular}{|l|l|l|l|l|l|}
    \hline
       & Not & Low & Medium & High & Professional \\ \hline
       code & 1 & 2 & 3 & 4 & 5 \\ \hline
        No. Respondents & 8 & 22 & 71 & 40 & 12 \\ \hline
    \end{tabular}
    \caption{Interest Level of Subjects in Sea Level Extremes}
    \label{interest_level_table}
\end{table}

As can be seen above the majority have medium level interest in sea level extremes. The spread of results to this question is approximatley normal, which could be seen as either that a good spread of subjects were contacted. However whether this does correctly model Trondheim's population level of interest in the subject is not very likely.  Survey participants were self selecting so that subjects have volunteered their time even with low or no interest in the research is helpful, but it is to be expected that people with an interest in a subject are more likely to take part in a survey.

This has been attempted to be minimised by making the survey as short and as easy to do as possible and by utilizing the researchers network. For more discussion on the careful utilisation of the researchers network to reach subjects who could have been missed find it under the section Survey Access. 

\section{Citizen Science}
Reference: Robinson et al (2012) Guide to citizen science: developing, implementing and evaluating citizen science to study biodiversity and the environment in the UK 
Why citizen science key for this project?
•	Resilience is assumed to be dependent on knowledge/awareness of potential population impacted by disaster/event
•	By including citizens and making them start to think about changing resilience of place to SLE’s may actually improve resilience
•	Need to consider a broad sector of societies awareness
•	Results driven by societal need (as highlighted by UN SDGs) hence moral allowance / legitimacy to ask for help / peoples time

The decision to utilise citizen science is driven by societal need, specifically whether this fulfils policy needs or community needs. for the reasons outlined in the introduction researching resilience to SLE requires knowledge of the awareness that the population has of the risk/vulnerability. Furthermore the gathering of this information has the potential to positively impact the resilience level by improving the awareness of the mentioned vulnerability. 

\section{Survey Access}

\begin{table}[!ht]
    \centering
    \begin{tabular}{|l|l|l|l|l|l|l|}
    \hline
     Access type & poster & email & social & via organisation  & via place of  & personal connection \\ \newline
       &  & & media & n membership & employment & to researcher \\ \hline
       code & 1 & 2 & 3 & 4 & 5 & 6 \\ \hline
      No. Respondents & 70 & 2 & 48 & 2 & 4 & 24 \\ \hline
    \end{tabular}
      \caption{Access to Survey}
      \label{access_survey}
\end{table}

\section{Future Research}

\section{Summary of Discussion of Framework - Lessons Learned}
%target marine workers earlier in process
%posters seem to be best method of accessing, followed by social media
%accessing at different season may impact results
%improve question phrasing so people dont just tick one box


The likelihood of their being another sea level extreme event in Trondheim before anyone gets the chance to repeat this research is significant. If this research was to be repeated it would be important to add any new sea level extreme events to the question on do you remember these events. 