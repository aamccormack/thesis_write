
\chapter{Discussion of Framework for Determining Resilience}
The overall objective of this project was to assist the creation of a framework for cheaply and quickly determining the resilience of place. The sections here discuss the lessons learnt from this project with a view to repetition of this type of research.

\section{Software Consideration}
The ease of providing data was a key design factor considered when designing the data collection method for this project. The goal was to create an online survey which could be done by the majority of participants in under 5 minutes, without missing out on potentially valuable data. This required a balancing act of maximising potential data gathered from a single participant against maximising number of participants. To do this several iterations were created and trialled as outlined in the method section.
\paragraph{}
Due to ease of later analysis, ease of repeatability, accessibility,  avoidance of disease spread and other major benefits an online survey method was chosen. The major downside of this technique is that it could exclude certain sectors of society due to lack of access to the internet. However, in Norway over 99 percent of the population have access to the internet (ssb https://www.ssb.no/en/statbank/table/11000/tableViewLayout1/ ) 
\paragraph{}
Once online surveys were the chosen method then which software was most appropriate needed to be considered. To do so many softwares were explored and then basic surveys were created and then tested on desktops, laptops, tablets, IOS smartphones and Android Smartphones on the perceived best candidates. The results of this exploration are outlined on table 7.1.

\begin{table}[h]
    \centering
    \begin{tabular}{|l|l|l|l|l|}
    \hline
        \textbf{software} & \textbf{nettskjema} & \textbf{Google forms} & \textbf{Arcgis survey 123} & \textbf{Survey Monkey} \\ \hline
        ~ & ~ & ~ & ~ & ~ \\ \hline
        Degree of security & high & high & medium & Medium \\ \hline
        Degree of privacy & high & medium & medium & Medium \\ \hline
        Link strength & High & medium & low & low \\ \hline
        Range of question formats & medium & medium & low & high \\ \hline
        Ease of use & medium & high & medium & high \\ \hline
        Gathers GIS data & no & no & yes & no \\ \hline
        Can upload pictures & no & yes & no & yes \\ \hline
        Can include several images & yes & yes & no & yes \\ \hline
        No. Surveys allowed & Unlimited & unlimited & Unlimited & limited \\ \hline
        No. questions allowed & unlimited & unlimited & unlimited & limited \\ \newline
        &  &  &  but slow >10 &  \\ \hline
        Download as Excel/CSV & yes & yes & yes & yes \\ \hline
        Notes & New to researcher & Subjects  & Slow  & Limited services\\ \newline
         & Norway & Require  & to load  & on free \\ \newline
         & specific & google account & on mobiles & subscription \\ \hline
    \end{tabular}
    \caption{Software Considerations}
    \label{table: software-considerations}
\end{table}

After this ArcGIS survey 123 was the chosen method of collection. The plan being the participants could draw upon an uploaded map where they believed the high-tide, current storm surge, past storm surge and future storm surges would be. This was believed to be the best technique as it was felt it would influence the participants answers least and allow for interesting GIS analysis later. 
\paragraph{}
However, difficulties arouse due to ArcGIS survey 123 struggling with more than one map being used in a single survey and difficulties of inputting this information on a smartphone. For this reason, a pilot survey was created in both ArcGIS survey 123 and Nettskjema. While ArcGIS survey 123 allowed for input of GIS data, Nettskjema does not. So, to compare survey methods maps of one of the selected areas were to allow for similar data input as can be seen in the sections below. 


\section{Pilot Survey and Focus Group Impacts to Project}
The results of the softwar trials, focus group and pilot survey were that the method for determining awareness was reasonable, but that a combination of numerical and image based visualisation of SLEs would be more useful, rather than map based. This decision did lose the spatial aspect of communication, but allowed for greater emotional connection.
\paragraph{}

One benefit which is missing when choosing to utilise simulated pictures of water rise rather than maps is that maps better show the area impacted. By looking from a bird’s eye view the scale and wide reach of SLEs impact is more obvious. However, this viewpoint is not the view which is normally experienced by those situated within these places. On the other hand, using simulated images allows for the interpretation to be easier and to potentially have more emotional impact as it is closer to the participants lived experiences. It is easier to look at a picture and realise that is where the water is in comparison to oneself. Greater psychological distancing can occur with the use of maps than photographs . However, this is only possible with a certain level of realness with the simulated image. Image editing is a slow process, particularly when aiming for realism, for this reason a full explanation of how the simulated images were created is given in the methods. 
\paragraph{}

When redesigning the survey after the Pilot Survey and Focus Group a greater focus was made on making sure the final survey was as accessible as possible. The new survey was tested on several participants before it was shared wider. The time taken to complete the new survey was measured on four participants, including those with mild visual impairments, dyslexia and those who were not using their native tongue. Each of these participants took under six minutes to complete the survey, hence fulfilling the design choice that the majority should be able to complete the survey in under five minutes. 
\paragraph{}



\section{Demographics}
The purpose of this research is to assist in the creation of a framework to determine resilience of a place, specifically to SLEs. This is to allow repeated measurements of resilience to determine whether a place is making improvements in its resilience. Resilience and vulnerability are impacted by demographic variables (\cite{rod_integrated_2012}), but improving of resilience shouldn't be done by changing population demographics. Arguing a place is resilient simply due to their being a high percentage of women, or immigrants, or the age of the population is not a useful nor moral lens in which to discuss improvements to resilience. It can be useful to highlight areas or groups who are particularly vulnerable, but not forcing those groups to disperse against their will. For this reason standard demographic questions were not part of this survey. Subjects attributes which local governance could reasonably, cost effectively and most importantly morally impact were prioritised as research variables over demographics. 
\paragraph{}
Furthermore by asking questions upon gender and age can influence how subjects answer these questions.*citation needed*

Awareness or a hazard is not determined by gender, even if gender can correlate with it.  Gender was determined not a key variable from literature review before the creation of the survey. This is the same for the majority of demographic variables which were not investigated during this research.
\paragraph{}
Certain demographic questions can be answered by inferring from the results to key questions. For example language skills and even immigration status can be inferred from the subjects decision to complete the survey in Norwegian or English. However these should not be focused on as it is known by the researcher that certain subjects chose to answer in the language which was not their native tongue. The lack of availability of the survey in other languages is a limiting factor. Percentage of the population who do not have a reasonable understanding of either Norwegian or English in Trondheim is ****. For this reason plus limitations of funds, researcher skills and time it was deemed acceptable for the posters, emails, social media, website and surveys to only be available in Norwegian and English. However, if this was to be repeated in other locations or nationally this would need to be reconsidered. 
\paragraph{}
 However the subjects reached truly represent Trondheim's stakeholders for SLEs can be debated. For example, survey participants were self selecting so that subjects with higher interest levels in SLEs were more likely to take part in a survey. There is also a strong skew towards subjects who are concerned about climate change. Attempts to reach as wide a pool of stakeholders were made. Encouraging them to participate was done by utilising the communication guidelines, making the survey as short and as easy to do as possible and by utilizing the researchers network. 



\section{Limitations of single-risk, single-scale risk analysis}
There is an increasing demand for multi-risk, multi-scale, multi-stakeholder determination of risk and resilience \cite{gerkensmeier_governing_2018} and \cite{cutter_community_2020}. An attempt has been made to display a method for quickly and cheaply determining social resilience utilisng the views of many stakeholders. There is limitations to this results of Trondheim being considered resilient as it is dependent only on the risk from SLEs. As highlighted by the subjects, there are other risks including risk of landslides. Landslide due to quickclay are also impacted by weather conditions, much like SLEs. 
\paragraph{}

While Trondheim is considered to currently have projected resilience for SLEs for the period of 2022 to 2050 and 2050 to 2100, this does not mean that normality will quickly be returned to if multiple other incidents occurred at the same time. These other risks could be natural hazards or caused by human choice. 

\section{Limitations of Technique}
The COVID-19 pandemic provided significant limitations to this research. This was the main factor in the decision to only use surveys and not complement with interviews. Furthermore, resilience is dynamic and will have been affected by the pandemic. 
\paragraph{}
The data collection limitations include the number of subjects and the impact of only surveying during the summer. The lack of marine workers within the subjects is a major limitation of this study, but does not prevent an overview of local knowledge. Particularly important is the limitation of only having the results for one year, which creates a snapshot of resilience rather than considering how it is changing overtime.  The exclusion of Nyhavvna as a research site is a limitation, which could be corrected in a repetition of this research.  The final limitation is the analysis technique which was limited due to the skewed distribution of the survey results.

%LIMITS OF EDITED PHOTOS
% HOW TO CREATE A FRAMEWORK
%LIMITS OF RESULTS - E.G. CODING /STATS LIMITS
%LIMITS OF SURVEYING AS TECHNIUE 
%LACK OF INTERVIEWS
%PEOPLE ONLY TICK ONE BOX WHEN ASKED CERTAINS QS E.G. COMMUNITY MEMEBERSHIP
%LESSONS LEARNED FOR REPEATING


\section{Future Research}
If repeating this project the researcher would change several aspects. The first aspect is making sure that for questions where subjects could select several answers they were made clear that it was requested that they select all the apply to them. The research would also either chose a different term than resident, or make it clearer what was meant by this term when collecting the data. 
\paragraph{}
In terms of reaching stakeholders this could have been improved by sending emails earlier. The results of survey access are likely impacted by this projects data collection having been conducted in the summer. An example email is included in the appendix and it is recognised that perhaps this could have been designed better. On the subject of survey access, it would be interesting to repeat the posters, but set it up so that it is clear which poster location subjects accessed the surveys via. The link between the subjects location and their response could be interesting; to do this the privacy of the subjects would need to be considered. The understanding of how subjects viewed place could be expanded upon. Perhaps another focus group on this subject could be conducted, or a stakeholder workshop. 
\paragraph{}

Finally the likelihood of there being another SLE in Trondheim before anyone gets the chance to repeat this research is significant. If this research was to be repeated it would be important to add any new sea level extreme events to the question on do you remember these events. 


