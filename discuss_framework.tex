
\chapter{Discussion of Framework for Determining Resilience}\label{discussion-framework}

The overall objective of this thesis is to contribute to the creation of an easily accessible framework able to inexpensively determine the resilience of a place to natural hazards. This chapter discusses the lessons learnt from this thesis with the perspective of reproducibility of the study, including the use of the presented framework in different areas of interest.

\section{Software Consideration}
Making it easy for stakeholders to take part in the research was a key design factor considered when designing the data collection method for this project. The goal was to create an online survey which could be completed by most participants in less than 5 minutes, without missing out on potentially valuable data. This required a balancing act between maximising potential data gathered from a single participant and maximising the number of participants. To do this, several iterations were created and trialled as outlined earlier (section \ref{method-pilot}).
\paragraph{}
An online survey method was chosen due to the method's ease of data codification, repeatability and accessibility. Restrictions related to the Covid-19 pandemic were also necessarily taken into consideration when designing the survey. The major downside of online surveys is that they can exclude certain sectors of society due to lack of access to the internet. However, in Norway 99 \% of the population between the ages of 16 and 79 regularly access the internet (\cite{walther-zhang_ict_2022}). 
\paragraph{}
The most appropriate software was then considered.  Various software packages were explored and then short surveys were created and  tested on desktops, laptops, tablets, IOS smartphones and Android Smartphones to determine the best option. The results of this exploration are outlined on table \ref{table: software-considerations}.

\begin{table}[h]
    \centering
    \begin{tabular}{|l|l|l|l|l|}
    \hline
        \textbf{Software} & \textbf{Nettskjema} & \textbf{Google Forms} & \textbf{ArcGIS Survey123} & \textbf{Survey Monkey} \\ \hline
        ~ & ~ & ~ & ~ & ~ \\ \hline
        Degree of Security & high & high & medium & medium \\ \hline
        Degree of Privacy & high & medium & medium & medium \\ \hline
        Link Strength & High & medium & low & low \\ \hline
        Range of Question Formats & medium & medium & low & high \\ \hline
        Ease of Use & medium & high & medium & high \\ \hline
        Gathers GIS Data & no & no & yes & no \\ \hline
        Can Upload Pictures & no & yes & no & yes \\ \hline
        Can Include Several Images & yes & yes & no & yes \\ \hline
        No. Surveys Allowed & unlimited & unlimited & unlimited & limited \\ \hline
        No. Questions Allowed & unlimited & unlimited & unlimited & limited \\ \newline
        &  &  &  but slow >10 &  \\ \hline
        Download as Excel/CSV & yes & yes & yes & yes \\ \hline
        Notes & New to researcher & Subjects  & Slow  & Limited services\\ \newline
         & Norway & Require  & to load  & on free \\ \newline
         & specific & google account & on mobiles & subscription \\ \hline
    \end{tabular}
    \caption{Software Considerations}{Nettskjema, Google Forms, ArcGIS Survey123 and Survery Monkey were considered. No. stands for the number of. }
    \label{table: software-considerations}
\end{table}

Due to its ability to gather GIS data, the ArcGIS Survey123 was the original software choice. The plan was that participants could draw directly on an map the reach of SLEs. This was considered as the best technique as it would have less influence on answers and allow for interesting GIS analysis. 
\paragraph{}

However, difficulties arose as this software struggled with more than one map being used in a single survey. There were also difficulties of entering this information on a smartphone. For this reason, a comparative pilot survey was created to compare the usability of ArcGIS Survey123 with Nettskjema. While ArcGIS Survey123 allowed for input of GIS data, Nettskjema does not.
\paragraph{}
Nettskjema was ultimately chosen due to its reliability, security, link strength and range of question formats.  This decision did minimise the spatial aspect of investigation and decreased the variety in possible responses.

\section{Pilot Survey and Focus Group Impacts to Project}
 
\paragraph{}
Results from the pilot survey and focus group were used to decide how information on SLEs were best communicated - maps, edited photographs or numerical values (section \ref{result-pilot-focus}). The potential impact on risk perception  including psychological distancing that may occur with the use of maps or edited photographs is thus far largely untested, especially when considering the risk from climate change (\cite{retchless_understanding_2018};  \cite{spence_psychological_2012}). Maps show the spatial extent of flood impacts more clearly than the edited photographs, however, using simulated images allows for the interpretation to be easier and to potentially have more emotional impact; for many people it is easier to place themselves within the context of a photograph of an known location. One caveat is that the photographs require a semblance of reality. Image editing is a slow process, particularly when aiming for realism.
\paragraph{}

When redesigning the survey after the Pilot Survey and Focus Group, a greater focus was on accessibility. The new survey was tested on several participants with mild visual and language difficulties (dyslexia, less familiarity with the language, astigmatism).  Each of these participants took under six minutes to complete the survey, with an average of five minutes hence fulfilling the design choice that the majority should be able to complete the survey in under five minutes. 
\paragraph{}



\section{Demographics}
Resilience and vulnerability are impacted by demographic variables (\cite{rod_integrated_2012}), but there are major ethical and practical implications to improving resilience by changing population demographics. Arguing a place is not resilient because of its demographic character is neither a useful nor moral lens through which to discuss resilience. It can be worthwhile to highlight areas or groups who are particularly vulnerable in order to provide them with extra support, but for the reasons outlined above, among others, standard demographic questions were not part of the survey. 


\paragraph{}
Whether the subjects who participated in the survey truly represent Trondheim's stakeholders for SLEs can be debated. For example, survey participants were self selecting so those with higher interest levels in SLEs were more likely to participate. There is also a strong skew towards subjects who are concerned about climate change. Attempts to reach as wide a pool of stakeholders were made. People who live, commute and work in coastal Trondheim were successfully included. Encouraging stakeholders to participate was done by utilising the communication guidelines (section \ref{com-design}), making the survey as short and as easy to do as possible and by utilizing the researcher's network. 



\section{Limitations of Single-Risk, Single-Scale Risk Analysis} \label{discussion-limitation-single-risk}
There is an increasing demand for a multi-risk, multi-scale, multi-stakeholder determination of risk and resilience (\cite{gerkensmeier_governing_2018}; \cite{cutter_community_2020}).  Only the risk from SLEs is investigated here. However, the same weather conditions can influence multiple natural hazards. While Trondheim is considered to currently have high resilience for SLEs for the period of 2022 to 2050 and 2050 to 2100, this does not mean that normality will quickly be returned to if  other natural hazards were to occur concurrently with a SLE. How best to include all hazards in a determination of resilience should be looked at in future research. The need to acquire a quick and inexpensive overview of resilience in an area that allows for repeated measurements and analysis of trends should be balanced against the need to consider all forms of hazards. 
\paragraph{}


\section{Summary of Limitations of Technique}
The COVID-19 pandemic provided imposed significant limitations on the research presented in this thesis and was a factor in the decision to only use surveys and not complement with interviews. It should also be noted that resilience is considered dynamic and will have been affected by the pandemic. 
\paragraph{}
The data collection limitations include the number of responses and the impact of only surveying during the summer. The lack of identified marine workers within the subjects is another limitation of this study, although it does not prevent an overview of local knowledge of SLEs. Of more importance, is the limitation of only having the results for one year, which creates a snapshot of resilience, rather than considering how it is changing over time. 


\section{Additional Considerations}
If reproducing this research, several modifications should be considered. Firstly, improvements to the clarity of the survey. Where subjects could select several answers, instructions need to be clearer. As discussed in section \ref{discuss-comm-membership}, there was a lack of participants identifying as more than one community member. The small number of students who also selected resident may reflect a tendency to only select one answer. However, it could also reflect the varied conceptualisation of what it means to be a resident. In future surveys, the term, resident, could either be defined more clearly within the survey or changed to a term with a more standardised conceptualisation.  
\paragraph{}
As was outlined in the Methodology (section \ref{data-collection}), the city of Trondheim was selected due to the ability for the author to utilise their personal network and due to restrictions related to the COVID-19 pandemic.  Nyhavna was not selected due to smaller population throughput at the time of data collection and as the major impact on SLEs in this area is ongoing construction and related subsidence (\cite{miljoenheten_og_byplankontoret_trondheim_kommune_9-notat-om-havnivastigning-og-stormflo---hensyn-i-arealplanlegging-nyhavnapdf_2020}), however future research should reconsider this site. Replication of this project in other Norwegian settlements with different past, present, and future SLE conditions and projections, would also be interesting. 

\paragraph{}
It may be interesting to set up the data collection so responses are connected to poster location, thus allowing the research to expand upon subjects' sense of place. The privacy of the subjects would need to be carefully managed using this technique. 
\paragraph{}
It is predicted that future researchers may try to utilise artificial intelligence (AI) to determine resilience. It can be argued that AI struggles with multiple datasets and prioritisation of data and it is expensive (\cite{shane_you_2019}). Currently, more well-suited technology exists, such as GIS, online surveys and dynamic documents. That all of the subjects responses in this research could be utilised with minimal re-coding was due to the careful design of the data collection. This technique is much quicker than paper surveys or stakeholder workshops and avoids transcription errors.
\paragraph{}
If this research were to be repeated in a rural environment the temporal variability of storm surges and land use characteristics should be considered; such factors appear to have minimal impact in urban and semi-urban environments and were not considered in this thesis (\cite{hoffken_effects_2020}). If making major changes to the survey format trialling these changes with a focus group is suggested. Finally, if this research was to be repeated, it would be important to add any additional, recent SLE events to the questions. 



