%This is the Summary
%%=========================================
%Here you give a summary of your your work and your results. This is like a management summary and should be written in a clear and easy language, without many difficult terms and without abbreviations. Everything you present here must be treated in more detail in the main report. You should not give any references to the report in the summary -- just explain what you have done and what you have found out. The Summary and Conclusions should be no more than two pages.

%Thus far resilience research often focuses on highly vulnerable locations, hence working within the supposedly less vulnerable Trondheim can help create a reference point for aimed place-based resilience

\chapter{Conclusion}

Sea level extremes are becoming more frequent globally and as discussed much of Trondheim's infrastructure and population is situated within the coastal zone. To understand Trondheim's changing resilience to sea level extremes, its technological, natural and social systems resilience were considered. Literature reviews, investigation of the coastline, analysis of city plans and planning permissions were used to determine natural and technological systems resilience. Online surveys of self-selecting stakeholders were used to aid the determination of Trondheim's social system resilience, specifically investigating the key variable of awareness. Four places within Trondheim - Skansen, Grillstad, Brattøra and Nidelva - were selected to investigate the city's resilience. They were chosen due to their higher physical vulnerability to SLEs and large daily throughput of people.
\paragraph{}
This thesis determines Trondheim's projected resilience to SLEs during 2022-2050 as high and the projected resilience for 2050-2100 as very high. This changing resilience is due to strong projected increase in technological system resilience while social system and natural systems have a slight reduction in projected resilience for this time period. Furthermore, stakeholders were deemed somewhat aware about changes to SLEs. The factors deemed to have impact on stakeholders' awareness of SLEs are whether they are residents, or whether they utilise the sources of family, newspapers and formal education for their information on climate change. This result enforces the the importance of framing the textual and visual communication for the designated audience as outlined in the communication design section of the methods.  The method used to collect data on awareness was deemed appropriate. There is room for improvement in the data analysis technique of determining awareness. Neither subjects self-ranked interest level, nor the place they chose to respond upon, were factors which were deemed to impact awareness.
\paragraph{}

Whether the surveys utilising visual simulations of sea level extremes to determine Trondheim's social system resilience to SLEs can also be used to improve awareness, hence improving resilience, would be an interesting avenue for future research. Improving resilience is a requirement of the Sendai framework and the United Nations Sustainable Development Goals 11 and 13. This thesis has shown that carefully designed surveys may be a useful tool in a framework designed to repeatedly measure resilience to see whether these goals are being met.


