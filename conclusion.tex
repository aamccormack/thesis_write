%This is the Summary
%%=========================================
%Here you give a summary of your your work and your results. This is like a management summary and should be written in a clear and easy language, without many difficult terms and without abbreviations. Everything you present here must be treated in more detail in the main report. You should not give any references to the report in the summary -- just explain what you have done and what you have found out. The Summary and Conclusions should be no more than two pages.



\chapter{Conclusion}

It is generally accepted that sea level extremes are becoming more frequent globally thus far resilience research often focuses on, highly vulnerable locations. However looking more closely at an area which is regarded as less can help create a reference point or target for a locations place-based resilience.

Much of Trondheim's infrastructure and population is situated within the coastal zone (figure \ref{fig:research_site}), as is common in Norway. To understand Trondheim's changing resilience to sea level extremes, its technological, natural and social systems resilience were considered.  Projected resilience was viewed as a projected outcome, which can be determined by analysing the systems which impact the risk (section \ref{theory-resilience}). In this case social, natural and technological, systems impacting the risk of sea level extremes. 
\paragraph{}

Literature reviews, investigation of the coastline, analysis of city plans and planning permissions were used to determine Trondheim's natural and technological systems resilience (section \ref{data-sources}). Online surveys of stakeholders were used to aid the determination of Trondheim's social system resilience, specifically investigating the key variable of awareness (section \ref{data-collection}). Awareness is an important theme in this thesis falling under the broader concept of local knowledge, which in turn falls under the concept of social system resilience (section \ref{theory-resilience}). A place-based understanding of community was used in this conceptualisation of social system resilience due to its dominance in Norwegian politics (section \ref{theory-resilience}). This framework was designed based on feedback received from a focus group and a pilot study that carefully considered the framing of textual and visual communication, particularly the use of edited photographs to display likely future scenarios of sea level extremes in Trondheim (section \ref{data-collection}). Four places within Trondheim - Skansen, Grillstad, Brattøra and Nidelva - were selected as study areas to investigate the city's resilience. They were chosen after after naturalistic observation due to their higher physical vulnerability to SLEs and large daily throughput of people which included a broad representation of society.

\paragraph{}
This thesis determines Trondheim's projected resilience to SLEs in the 2022-2050 period as high (section \ref{RQ1-findings}), while the projected resilience for 2050-2100 is determined to be very high (section \ref{RQ2 - findings}). These high values of projected resilience are mainly due to a the large improvement of the technological systems outlined in the cities plans and building requirements (\ref{tech-resilience-discussion}), while the social and the natural systems show a slight reduction  in the projected resilience for the same time period (section \ref{RQ2 - findings}). Resilience is highly dynamic and over the extended time frame considered in this thesis (2022-2100) these results are likely to change. Hence the requirement for designing a framework for repeated measurements to determine the trend of a places resilience as was attempted here. By considering the trend we can determine whether Trondheim is inline with the United Nations Sustainable Development Goals (UN SDG’s) 11 and 13 requirement of improving resilience of human settlements.

\paragraph{}
Stakeholders were deemed somewhat aware about changes to SLEs (section \ref{RQ3 - finding}). The factors deemed to have impact on stakeholders' awareness of SLEs are whether they are residents, or whether they utilise the sources of family, newspapers and formal education for their information on climate change. This result enforces the importance of framing the textual and visual communication for the designated audience as outlined in the communication design section of the methods as this can impact awareness and perception.  The method used to collect data on awareness was deemed appropriate. There is room for improvement in the data analysis technique of determining awareness. Neither subjects self-ranked interest level, nor the place they chose to respond upon, were factors which were deemed to impact awareness.
\paragraph{}



Whether the surveys utilising visual simulations of sea level extremes to determine Trondheim's social system resilience to SLEs can also be used to improve awareness, hence improving resilience, would be an interesting avenue for future research. As is the impact on the research of including sectors of society who are often non-deliberately excluded from such research, for example by only utilising volunteer based stakeholder workshops, but can be the most vulnerable including migrants, short term residents, tourists, youth and the immune compromised. Improving resilience is a requirement of the Sendai framework and the United Nations Sustainable Development Goals 11 and 13. This thesis has shown that carefully designed surveys may be a useful tool in a framework designed to repeatedly measure resilience to see whether these goals are being met.


