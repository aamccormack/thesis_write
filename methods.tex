% Methodology section including data collection and data analysis technique
%no results or discussion here



\chapter{Methodology}
\section{Pilot Survey and Focus Group}

\section{Visulisation of Risk}
I.E. WHY PHOTOS NOT NUMBERS NOT MAPS 

\section{Data Collection}

Data collection for this study comprised of two aspects.  The first involved literature reviews and collecting models of sea level rise and sea level extremes. The aim was to create a realistic visualisation of sea level extremes in Trondheim between 1950 and 2100 to utilise in discussion of awareness of sea level extremes. 
    \paragraph{}
The second was an online survey conducted in the summer of 2022. The aim was to explore subjects experience, awareness and information access about sea level extremes and climate change. Subjects were recruited via social media, email and posters placed along Trondheim’s coastline focusing on locations where people wait. The survey was designed to take under five minutes to maximise responses. Stakeholders targeted were those with direct experience of Trondheim’s coast specifically the four places – Brattøra, Skansen, Nidelva and Grillstad. These locations were chosen after naturalistic observation along all populated areas of Trondheim’s coast. Trondheim was selected due to the potential to utilise personal network plus COVID-19 restrictions. These locations were chosen due to their high throughput of all demographics and perceived physical vulnerability. Nyhavana was not selected due to less population throughput at the time of data collection and as the major impact on sea level extremes is ongoing construction and related subsidence \cite{miljoenheten_og_byplankontoret_trondheim_kommune_9-notat-om-havnivastigning-og-stormflo---hensyn-i-arealplanlegging-nyhavnapdf_2020}


\paragraph{}
Brattøra, Skansen and Nidelva were highlighted during case presentations by the municipality as areas which can expect a risk of flooding, but that the water will then disappear again afterwards. This means they have special building requirements \cite{hanssen_saksframlegg_2013}. Grillstad was not specified in this report as it was built after, but its location is in the area requiring these special requirements so it can also be considered a biophysically vulnerable area.  

 
\section{Data Analysis}
 
\paragraph{}
  The survey comprised of 26 questions addressing awareness, memory of sea level extremes, interest levels in sea level extremes, community membership, information access, length of residence, attitudes as well as space to highlight other place-based risks. 153 responses were collected. 30 percent of respondents had no memory of sea level extremes in Trondheim with 70 percent having memory of one or more event. A third (50/153) remembered the most recent sea level extreme in February 2020. The survey data were analysed using a linear regression as awareness is considered continuous. The survey data were codified and exported to R, which was used to analyse all quantitative data. 

  Participant diversity is inherently biased when relying on goodwill, but a wide range of levels of interest in sea level extremes and community membership were surveyed including those who were not interested and who had limited knowledge of Trondheim’s coasts, as can be seen in the results.
  \paragraph{}

**if variables turn out to be dependent will switch to linear mixed effects modelling.  


\section{Limitations}
covid-19 ->>> no interviews
change in who is out and about
change is feelings toward risk

researchers health impacts meant timelines couldnt be idealised
e.g. summer aint the best time



