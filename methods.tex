% Methodology section including data collection and data analysis technique
%no results or discussion here



\chapter{Methodology}
\section{Pilot Survey and Focus Group}

\section{Visulisation of Risk}
I.E. PHOTOS NOT NUMBERS NOT MAPS 
how i made the photos

To determine awareness five questions were asked, of these three included four pictures of simulated sea level extremes for each of the research sites. These were created based of models from...



\begin{table}[h]
    \centering
    \begin{tabular}{|l|l|}
        \hline
     	Research Site & Photo water level (cm) \\ \hline
            Grillstad & 50 \\ \hline
            Skansen & -23 \\ \hline
            Bakklandet & -159 \\ \hline
            Brattøra	& -42 \\ \hline
    \end{tabular}
    \caption{Water Levels on day of Photo Taken for Simulated SLE}
    \label{tab:water_level_photo}
\end{table}


words words words

\begin{table}[h]
    \centering
    \begin{tabular}{|l|l|l|l|l|}
    \hline
        2022 & 2022 & 2022 & 2022 & 2022 \\ \hline
        mean high water neeps & mean high water springs & 20yr storm surge  & 200 yr storm surge  & 1000 yr storm surge  \\ \hline
        55 & 119 & 216 & 234 & 244 \\ \hline
    \end{tabular}
    \caption{Sea Level Extremes Projections}
    \label{2022_sle_projections}
\end{table}

words words words
\begin{table}[h]
    \centering
    \begin{tabular}{|l|l|l|l|}
    \hline
        2090 & 2090 & 2090 & 2090 \\ \hline
        mean high water springs & 20yr storm surge  & 200 yr storm surge  &  1000 yr storm surge  \\ \hline
        172 & 269 & 286 & 297 \\ \hline
    \end{tabular}
    \caption{Sea Level Extremes Projections}
    \label{2090_sle_projections}
\end{table}
words words
\section{Data Collection}

Data collection for this study comprised of two aspects.  The first involved literature reviews and collecting models of sea level rise and sea level extremes. The aim was to create a realistic visualisation of sea level extremes in Trondheim between 1950 and 2100 to utilise in discussion of awareness of sea level extremes. 
    \paragraph{}
The second was an online survey conducted in the summer of 2022. The aim was to explore subjects experience, awareness and information access about sea level extremes and climate change. Subjects were recruited via social media, email and posters placed along Trondheim’s coastline focusing on locations where people wait. The survey was designed to take under five minutes to maximise responses. Stakeholders targeted were those with direct experience of Trondheim’s coast specifically the four places – Brattøra, Skansen, Nidelva and Grillstad. These locations were chosen after naturalistic observation along all populated areas of Trondheim’s coast. Trondheim was selected due to the potential to utilise personal network plus COVID-19 restrictions. These locations were chosen due to their high throughput of all demographics and perceived physical vulnerability. Nyhavana was not selected due to less population throughput at the time of data collection and as the major impact on sea level extremes is ongoing construction and related subsidence \cite{miljoenheten_og_byplankontoret_trondheim_kommune_9-notat-om-havnivastigning-og-stormflo---hensyn-i-arealplanlegging-nyhavnapdf_2020}


\paragraph{}
Brattøra, Skansen and Nidelva were highlighted during case presentations by the municipality as areas which can expect a risk of flooding, but that the water will then disappear again afterwards. This means they have special building requirements \cite{hanssen_saksframlegg_2013}. Grillstad was not specified in this report as it was built after, but its location is in the area requiring these special requirements so it can also be considered a biophysically vulnerable area.  

\section{Communication Design}
\paragraph{}
When utilising citizen science communication styles is very important. This section details the various methods used to reach subjects and encourage them to participate. Accessibility, trust and legitimacy are important attributes with all citizen science, but even more so in subjects with a potential emotional impact \cite{tweddle_guide_2012}. A simulated image which shows a place a participant cares about as being impacted by a sea level extreme could result in a strong emotional impact which if not handled carefully could turn the participant away from the survey and research on this in general.
\paragraph{}
The Web Accessibility Guidelines \cite{henry_web_2022} and Story Map Accessible Design Principles \cite{todd_liz_getting_nodate} were actively used during the creation of the website and online surveys. These guide good standard of visual communication and access by varied pieces of technology including e-readers. The guideline used for text  was the Principles for effective communication and public engagement on	climate change: A Handbook for IPCC authors \cite{corner_a_principles_2018}. While not written for citizen science its advice for when discussing climate change and how to connect with people on these subjects was very useful. 
\paragraph{}
The design principles which were followed can be summarised as:
\begin{itemize}
    \item Make it as easy as possible to participate
    \item Utilize personal brand to enhance connection and trust
    \item Be succinct
\end{itemize}
\paragraph{}
These were created from the guidelines mentioned above plus previous experience working in communications as well as training in visual communication, particularly cartography. To enhance legitimacy the results of this project will be shared to interested subjects via the website which was created to reach subjects after completion. 
\paragraph{}
In the appendix you can find example emails, social media posts and the poster used to access subjects. The full survey in Norwegian and English is also attached to demonstrate how the communication guidelines used were implemented.
\paragraph{}

Attempts were made to keep both the English and Norwegian survey as close as possible to allow for easy comparison. However direct translation is not always possible and the nuance and implication of word choices can have significant impact on the results. This was minimised by the researcher writing both surveys rather than relying on a translator and by getting it checked by several individuals with understanding of the topic,



\section{Data Analysis}
 The data analysis utilised excel and R studio and included Kruskal Wallis Rank Sum testing and linear modelling. 
\paragraph{}
  The survey comprised of 26 questions addressing awareness, memory of sea level extremes, interest levels in sea level extremes, community membership, information access, length of residence, attitudes as well as space to highlight other place-based risks. 153 responses were collected. 30 percent of respondents had no memory of sea level extremes in Trondheim with 70 percent having memory of one or more event. A third (50/153) remembered the most recent sea level extreme in February 2020. The survey data were analysed using a linear regression as awareness is considered continuous, however due to lack of linearity in original dotplots another method was also utilised. Kruskal wallis rank sum tests were used due to the lack of obvious linearity, as as awareness is considered continuous and as the data was not normally distributed. Logistic regression was considered, but was decided against as how you would split awareness into a dummy variable would have been the most significant impact on the results. The survey data was taken directly from Nettskjema, codified using excel and then exported to R, which was used to analyse all quantitative data. 
\paragraph{}
  Participant diversity is inherently biased when relying on goodwill, but a wide range of levels of interest in sea level extremes and community membership were surveyed including those who were not interested and who had limited knowledge of Trondheim’s coasts, as can be seen in the results. The original intention was to determine awareness as the ability to answer 5 questions. This would hopefully create a homoscedasticitic (i.e. the residuals have constant variance at every level of x) variable with independent and normally distributed residuals. However quick analysis of the data showed that this would not be the case.
\paragraph{}
  For example only 3 subjects answered correctly to the question "How much do you think the sea level has changed in the past 30 years?", creating a incredibly skewed distribution. For this reason this variable (slr-past) it has been excluded from the determination of awareness. This does not mean it is not considered in the answer to research question 2, but research question 3 requires a significant percentage being deemed aware. 
\paragraph{}
  A higher percentage *input value* answered correctly to the question "How much do you think the sea level will change in the next 30 years?". However this percentage was deemed insignificant and more due to luck as 2/7 answers were appropriate. For this reason this variable (slr-future) was also excluded for the determination of awareness as used to answer research question 3. However, it is used when answering research question 2. The other questions used to determine awareness utilised images with simulations of the sea level extreme, unlike these questions which solely used numeric values. This is another reason why only three questions were included in the determination of awareness in contrast with the original plan. The exclusion of these potential variables is discussed further in the discussion, particularly what it infers about the ability to visualise from a number. 
\paragraph{}
Awareness was calculated using the responses to the three questions "Which image shows the current 20-year storm surge?", "Which image shows the 20-year storm surge projected for 2090?" and "Which image displays the current high tide?". In the appendix the full survey in both Norwegian and English is attached, which shows how subjects received both numeric and visual representation of the sea level extremes when answering these questions. 



\section{Limitations}
The COVID-19 pandemic provided significant limitations to this research. Including the choice to only use surveys and not complement with interviews for this investigation of Trondheim's resilience to sea level extremes. Further more resilience is dynamic as is views on risk and the pandemic will have had impacts on both. The timeline of this as a masters project put limitations on the gathering of data, especially when considering the social system aspects of resilience. The data gathered limitations include the number of subjects and the impact of only surveying during the summer. The lack of marine workers within the subjects is a major limitation of this study, which should be corrected if repeated, but does not prevent an overview of local knowledge. Particularly important the limitation of only having the results for one year, which creates a snapshot of resilience rather than considering how it is changing. Furthermore the exclusion of Nyhavvna from the research sites which is explained in the discussion framework does provide a limitation.  The final limitation is the analysis technique which was limited due to the skewed distribution of the survey results.

