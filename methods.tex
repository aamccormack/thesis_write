\title{Method}
Data collection for this study comprised of two aspects.  The first involved literature reviews and collecting models of sea level rise and sea level extremes. The aim was to create a realistic visualisation of sea level extremes in Trondheim between 1950 and 2100 to utilise in discussion of awareness of sea level extremes. 
The second was an online survey conducted in the summer of 2022. The aim was to explore subjects experience, awareness and information access about sea level extremes and climate change. Subjects were recruited via social media, email and posters placed along Trondheim’s coastline focusing on locations where people wait. The survey was designed to take under five minutes to maximise responses. Stakeholders targeted were those with direct experience of Trondheim’s coast specifically the four places – Brattøra, Skansen, Nidelva and Grillstad. These locations were chosen after naturalistic observation along all populated areas of Trondheim’s coast. Trondheim was selected due to the potential of utilisation of researcher’s personal network plus COVID-19 restrictions. These locations were chosen due to their high throughput of all demographics and perceived physical vulnerability in models. Nyhavana was not selected due to less population throughput at the time of data collection and as the major impact on sea level extremes is ongoing construction and related subsidence (Miljøenheten og Byplankontoret, Trondheim kommune, 2020).
Brattøra, Skansen and Nidelva were highlighted during by the municipality as areas which one can expect a risk of flooding, but that the water will then disappear again afterwards, this means they need special requirements when planning building (Hanssen & Langedal, 2013). Grillstad was not specified in this report as it was built after, but its location is in the area requiring these special requirements so it can also be considered a biophysically vulnerable area.
 
 

Participant diversity is inherently biased when relying on goodwill, but a wide range of levels of interest in sea level extremes and community membership were surveyed including those who were not interested and who had limited knowledge of Trondheim’s coasts, as can be seen in figures ***. 
The survey comprised of 26 questions addressing awareness, memory of sea level extremes, interest levels in sea level extremes, community membership, information access, length of residence, attitudes as well as space to highlight other place-based risks. 153 responses were collected. 30% of respondents had no memory of sea level extremes in Trondheim with 70% having memory of one or more event. A third (50/153) remembered the most recent sea level extreme in February 2020. The survey data were analysed using a linear regression as awareness is considered continuous. The survey data were codified and exported to R, which was used to analyse all quantitative data. 
*if variables turn out to be dependent will switch to linear mixed effects modelling.




References to literature can be given in two different ways:
\begin{itemize}
\item As an \emph{explicit} reference: It is shown by \citet{lundteigen08} and partly also by \citet{rausand04}  that \ldots.
\item As an \emph{implicit} reference: It is shown \citep[e.g., see][Chap. 4]{rausand04} that \ldots.
\end{itemize}
In the example above, we have used ``author-year'' references, which is the preferred format. 
\begin{remark}
Following agreement with your supervisor, you may also refer by numbers, for example,  [1]. To do this, open the file \texttt{ramsstyle.sty} and  comment out (by \%) the command \texttt{$\backslash$usepackage\{natbib\}} and un-comment the corresponding command \texttt{$\backslash$usepackage[numbers]\{natbib\}}.\footnote{Notice the strange way we have to write the ``backslash'' in the text. This is because the ``backslash'' is a command in \LaTeX.}
\end{remark}
 You may include a link to the Internet in the text or in a footnote by using a command like: \url{http://www.ntnu.edu/ross}. 

When you refer to the scientific literature, you should always write in \emph{present} tense. Example: \citet{rausand04} show that \ldots.

\begin{remark}
Hyperlinks are included by the command \texttt{$\backslash$usepackage\{hyperref}\} in \texttt{ramsstyle.sty}. If you feel that the hyperlinks are disturbing when you enter the text, or want to avoid the hyperlinks in printed text, you may either comment out or edit this command in \texttt{ramsstyle.sty}.
\end{remark}
%%=========================================
\subsection*{What Remains to be Done?}
After you have defined and delimited your problem -- and presented the relevant results found in the literature within this field, you should sum up which parts of the problem that remain to be solved.