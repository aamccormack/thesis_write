\chapter{Background}

\section{Introduction to Research Area}

Locations of Project Background
Maps here
Places chosen to research 

Trondheim and SlEs
•	2.06 m high water (NN2000)
•	Maximum observed in Trondheim is 2.60m in 1971
•	City plans for 4.87m by 2100
•	More details on chosen research areas
•	Include maps here 
•	Social demographic stats
•	Risk level stats

\section{Research Sites}

\paragraph{}
\begin{table}[!ht]
    \centering
    \begin{tabular}{|l|l|l|l|l|}
    \hline
        location & Brattøra & Grillstad & Skansen  & Nidelva \\ \hline
        PV 70 years ago & high & high & medium & Low \\ \hline
        PV now &  medium &  medium &  low &  low \\ \hline
        PV in 70 years &  high &  high &  medium &  medium \\ \hline
        Dominant & Office space  & Residential & Recreational  & Residential and \\ \newline
        use & and harbour &  only   &  and industry & and commercial  \\ \hline
        Land type & Reclaimed land & Reclaimed land & Managed coastline  & Altered tidal river \\ \hline
        protection & Harbour wall & Harbour wall & Harbour wall, placed rocks & inland \\ \hline
    \end{tabular}
    \caption{Research Sites}
    \label{table:research-sites}
\end{table}

\section{Communication Design}
guidelines used for survey design

guidelines used for advertising

guidelines used for visual communication

guidelines used for communication about climate change

To maximise use of citizen science communication styles is very important including accessibility. In this section I will got into detail about the various methods I used to reach participants and encourage them to fill in my survey. Accessibility, trust and legitimacy are important attributes with all citizen science, but even more so in subjects with a potential emotional impact. A simulated image which shows a place a participant cares about as being impacted by a sea level extreme could result in a strong emotional impact which if not handled carefully could turn the participant away from the survey and research on this in general.


•	General design principles
o	 I used 2 sites 
o	https://storymaps.arcgis.com/collections/d34681ac0d1a417894a3a3d955c6913f?item=15 story map accessible design principles 
o	https://www.w3.org/WAI/standards-guidelines/wcag/ Web Content Accessibility Guidelines (WCAG)
o	Plus experience in coms & map design
o	Which highlights succinct is king
•	Communication style guideline used
Corner, A., Shaw, C. and Clarke, J. (2018). Principles for effective communication and public engagement on
o	climate change: A Handbook for IPCC authors. Oxford: Climate Outreach 
•	Website 
o	url shortener
o	accessibility
o	design for smartphones and laptop/desktops
o	easy to use
o	ux / ui design funnels you to the survey
o	professional 
o	personal brand utilisation to confer trust and legitimacy
•	Email
o	 Succinct is key
o	When appropriate utilisation of university email
	To provide legitimacy
	To avoid spam filters
o	Not used when I already had personal connection with targeted company
o	Who did I email 
•	Poster
o	Qr code
o	Nice pic which targets water users
•	Social media
o	Timing!
o	Videos on fb are prioritised by algorithm
o	So made loop of changing SLEs as gif and as video
o	Engagement rate
•	Utilisation of researcher contacts
•	Results from survey how did you access survey
o	Especially how many had personal connection to researcher
•	To maintain legitimacy will need to communicate the results of my research to participants after I am finished – will send an email to participants who requested more info and will post to website 


%decisions dont go here, but in the methods