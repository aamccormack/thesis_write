\chapter{Background}

\section{Norway and SLEs in the Present Day}
 It would be reasonable to consider Norway as dominated by its relationship with the sea. All of its major cities are located on the coast.  By January 2022, 31.6 percent of the Norway's coastal zone has been influenced by buildings, railways, roads and agriculture (\cite{engebakken_construction_2022}). This is notable as only 68.4 percent of the Norwegian coastline is currently deemed accessible due to its steep terrain (\cite{engebakken_construction_2022}). Norwegian weather is predicted to become more extreme (\cite{rod_integrated_2012}), this is in line with global trends which project increase in coastal flooding due to extreme weather and sea level rise (\cite{hoffken_effects_2020}). 
\paragraph{}

Due to changing climate and settlement patterns in Norway, there is a increase in the threat of natural hazards from the sea, which requires more attention given to regional vulnerability and the resilience of places (\cite{opach_seeking_2020};\cite{rod_three_2015}) . Quantitative community resilience measurement approaches have been tried, but there are several limiting factors and, most importantly, they rarely offer policy makers the strategies required for building resilience (\cite{opach_seeking_2020}; \cite{gerkensmeier_governing_2018}). Important information can be gained from vulnerability indices, including understanding of social vulnerability, but questions remain about how social vulnerability can be improved upon to build resilience. 


\section{Introduction to Research Area}

Trondheim is situated in a protected location, deep within Trondheimsfjord, which protects it from the North Sea (figure 2.1). Many assume that Trondheim is not at risk from SLEs due to this protected location, combined with its geological setting. This assumption is intensified by the understanding that Norwegian land is rising due to glacial melt. However, isostatic uplift is an incredibly varied process across the country and the basic model that the majority of the population has about sea level rise and land rise, may be creating a false confidence.  The maximum observed sea level extreme in Trondheim is 2.06m (NN2000), recorded in 1971 (\cite{tides_high_2022}). Trondheim municipality plans indicate that the city should be prepared for SLEs of 4.87m by 2100. 

\begin{figure}[h!]
    \centering
    \includegraphics[width=1.0\textwidth]{fig/Trondheimsfjord.png}
    \caption{The location of Trondheim - the city is situated in Trondheimsfjord, which provides protection to the North Sea. Figure created using ArcGIS Pro}
    \label{fig:research area Trondheim}
\end{figure}



\section{Research Sites}
Four sites which are situated on Trondheim's coast were chosen for their high daily population throughput, large amounts of infrastructure and coastal characteristics. These are Brattøra, Grillstad, Skansen and Nidelva. Physical vulnerability to SLEs is diverse for these sites, as can be seen in figure 2.2 below and is further outlined in table 2.1. 
\paragraph{}

\begin{figure} [h]
    \centering
    \includegraphics[width=1.0\textwidth]{fig/trondheim_research_sites_grey_circles.png}
    \caption{ The location of the Research Sites in Trondheim. The sites are highlighted with purple circles and are called Brattøra, Grillstad, Skansen and Nidelva. Figure was created using ArcGIS Pro.}
    \label{fig:research sites}
\end{figure}

\paragraph{}

Factors included in physical vulnerability are natural resistance to erosion, engineered resistance to erosion, engineered protection to SLEs, infrastructure directly upon coastline, settlement patterns, usage patterns and projected changes in SLEs. The physical vulnerability was determined from direct observation,  models from (\cite{kartverket_se_2020}) and consulting planning documents (TEK10 /17) for each site (\cite{miljoenheten_og_byplankontoret_trondheim_kommune_9-notat-om-havnivastigning-og-stormflo---hensyn-i-arealplanlegging-nyhavnapdf_2020}). 


\paragraph{}
\begin{table}[!ht]
    \centering
    \begin{tabular}{|l|l|l|l|l|}
    \hline
        \textbf{location} & \textbf{Brattøra} & \textbf{Grillstad} & \textbf{Skansen}  & \textbf{Nidelva} \\ \hline
        PV 70 years ago & high & high & medium & Low \\ \hline
        PV now &  medium &  medium &  low &  low \\ \hline
        PV in 70 years &  high &  high &  medium &  medium \\ \hline
        Dominant & Office space  & Residential & Recreational  & Residential and \\ \newline
        use & and harbour &  only   &  and industry & and commercial  \\ \hline
        Land type & Reclaimed land & Reclaimed land & Managed coastline  & Altered tidal river \\ \hline
        protection & Harbour wall & Harbour wall & Harbour wall, placed rocks & inland \\ \hline
    \end{tabular}
    \caption{The changing physical vulnerability to SLEs (PV) of the Research Sites.}
    \label{table:research-sites}
\end{table}

Brattøra is the research site with the least amount of residential population. The dominant use is as office space, which is protected by a harbour wall and a small harbour behind that. The location is upon reclaimed land and for this reason it had a very high physical vulnerability 70 years ago. The modern day harbour and area's design allow it to have medium physical vulnerability now, but it is projected to have several sections regularly flooded in 70 years. Currently, these areas are predominantly sustainable urban development schemes with footpaths and car parks where a lot of the flooding would occur, but there is still risk of more significant impacts from SLEs beyond nuisance flooding, such as shutting off of important transport routes and preventing access to offices.
\paragraph{}
Grillstad is also located on reclaimed land and again had a very high physical vulnerability 70 years ago. It is also located behind a harbour and harbour wall, but in contrast to  Brattøra, the dominant use is residential. There are several small commercial ventures, but the vast majority serve the needs only of local residents. There is no major transport connections or industry in this site. 
\paragraph{}
Nidelva is a less obvious choice for research into SLEs as it is situated further from the coastline. The land type here is altered tidal river and its physical vulnerability 70 years ago was low. The dominant uses are residential and commercial and it has lower physical vulnerability predicted in 70 years than Brattøra and Grillstad. Nidelva site has the added complication of the river level potentially impacting the height of the water at certain times. The river here is controlled by the dam in Leirfoss which is part of the hydroelectric power scheme which provides energy to the area.
\paragraph{}
Skansen has two dominant uses of recreation and industry. However, there is also significant residency in the area, the majority being high rise flats at least 10m from the coastline. While several aspects of this coastline may appear more natural, it is a managed coastline which is protected by placed rocks, a harbour wall and small bays. Unlike the other sites, there have been attempts to utilise natural techniques to prevent flooding in this area. This includes the restoration of Illabekken river and banks and the use of non-concreted areas which are covered in plants to improve infiltration (\cite{selliseth_ilabekken_2021}).
\paragraph{}

\paragraph{}
Brattøra, Skansen and Nidelva were highlighted during case presentations by the municipality as areas which can expect a risk of temporary flooding. This means they have special building requirements (\cite{hanssen_saksframlegg_2013}). Grillstad was not specified in this report as it was built after, but its location is in the area requiring these special requirements so it can also be considered a biophysically vulnerable area.  


