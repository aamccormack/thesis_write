\chapter{Results}

\section{Pilot Survey and Focus Group}

\subsection{Pilot Survey Results}
The pilot survey was conducted using 14 subjects on the 21st and then 25th of March, 2022. The subjects were classed as highly aware as they all had familiarity with the subject, they were either Natural Resources Masters students or members of Trondheim Kayak Klubb who had the project presented by the researcher in advance of the survey. They were asked seven questions and of them three of them were used to attempt to determine subjects awareness. Figures below display the maps used when asking about the projected heights of sea level extremes.

\begin{figure} [h]
    \centering
    \includegraphics[width=8cm]{fig/brattora question on 2022 high tide quadrant.png}
    \caption{Which image displays Brattøra's High tide? Image used in question from pilot survey. The answer which matched models from \cite{kartverket_se_2021} is B.}
    \label{fig:Brattora_2022_hightide}
\end{figure}

\begin{figure}[h]
    \centering
    \includegraphics[width=8cm]{fig/brattora question on 2090 20 yr storm surge quadrant.png} 
    \caption{Which image displays Brattøra's 20 year storm surge in 2090? Image used in question from pilot survey. The answer which matched models from \cite{kartverket_se_2021} is B.Caption}
    \label{fig:brattora_2090_stormsurge}
\end{figure}

\begin{figure}[h]
    \centering
    \includegraphics[width=8cm]{fig/brattora question on 2022 20 yr storm surge quadrant.png}
    \caption{Which image displays Brattøra's 20 year storm surge now? Image used in question from pilot survey. The answer which matched models from \cite{kartverket_se_2021} is B.Caption}
    \label{fig:brattora_2022_stormsurge}
\end{figure}

The next figure displays the results from the determination of awareness done on the pilot subjects.

\begin{figure}[h!]
    \centering
    \includegraphics{fig_results/pilot-survey-results.png}
    \caption{Pilot Survey Results. The answers which matches the model \cite{kartverket_se_2021} were always B. This was done for ease of speedy analysis and discussion after survey completion. As can be seen the majority chose answer B for question "which image display's Brattøra's 20-year storm surge now". However for "Which image display's Brattøra's 20-year storm surge in 2090?" the majority chose C, with D the next popular answer   }
    \label{fig:pilot_survey_results}
\end{figure}

As can be seen the pilot survey subjects did not easily get the correct answers for sea level extremes in Trondheim. The lack of answers which corresponded with models was of concern so a focus group was run. 

\newpage
\subsection{Focus Group Results}
The focus group used seven of the subjects from the pilot survey. The purpose of the focus group was to determine whether the subjects did have knowledge about sea level extremes and if so what prevented them from choosing the answer which corresponded with the models. Quickly the focus group highlighted that the maps were too similar. It was too difficult to tell the maps apart due to the relatively small changes. This was particularly the case for the question about High tide. Furthermore several participants highlighted that they did not think of sea level extremes in terms of area flooded, but solely in terms of changing height. This one dimensional view was highlighted as a limiting factor in the understanding of potential impacts. 
\paragraph{}

The focus group were shown the slide featured below. They were then asked which visualisation would help them answer the questions set in the pilot survey.
\begin{figure}[h!]
    \centering
    \includegraphics[width=1\textwidth]{fig_results/slide-pilot-survey.png}
    \caption{Slide shown to the focus group. The first image is how sea level extremes could be visualised using maps of the area. The second image is how sea level extremes could be visualised using photo editing of the waterline. The third image is how the sea level extremes could be visualised in a numeric way. This slide was used to spark discussion and allow for comparison.}
    \label{fig:slide}
\end{figure}

Several of the participants chose numeric value visualisation, but recognised that was likely due to their professional background. Those with a longer knowledge of Trondheim favoured the simulated sea level extremes pictures. This was also highlighted as the option with the highest emotional impact and spiked others interest more. The results of the focus group and pilot survey were that this method was reasonable for determining awareness but that a combination of numerical and image based visualisation of sea level extremes would be used, rather than map based. This decision did loose the spatial aspect of communication, but allowed for greater emotional connection.

\section{Technological Systems}

Distinguishing technological and natural systems can be difficult in a landscape which has been actively shaped by its population for so long. The interaction between each of the systems of which the interplay results in the projected resilience is significant. For Trondheim the location of infrastructure including buildings and roads is the primary focus of the technological system which impact resilience to sea level extremes. How this infrastructure is built is of course of great importance to whether a speedy return to normality is possible after a sea level extreme event. Yet even perfectly designed infrastructure will still be impacted due to flooding. From the obvious  prevention of use during flooding, to post event clean up and the wear and damage which can occur from sea level extreme events.
\paragraph{}
The table below displays the number of building which are likely to be impacted during different sea level extremes in Trondheim. This is modelled by \cite{kartverket_se_2021} and was the base of later water level simulations as used in this project. Importantly this modelling is more localised than previous models of sea level extremes in Norway.

\begin{table}[h]
    \centering
    \begin{tabular}{|l|l|l|l|l|}
    \hline
        water level & no. buildings  & ~ & ~ & ~ \\ \hline
        ~ & private & private & public  & critical  \\ \newline
        ~ & buildings & businesses & buildings & buildings \\ \hline        
        20 years return height now & 160 & 77 & 10 & 0 \\ \hline
        200 years return height now & 214 & 87 & 10 & 0 \\ \hline
        1000 years return height now & 242 & 104 & 14 & 0 \\ \hline
        Flooded 2090 & 66 & 51 & 8 & 0 \\ \hline
        20-years return height 2090 & 264 & 119 & 17 & 1 \\ \hline
        200-years return height  2090 & 308 & 136 & 24 & 1 \\ \hline
        1000-years return height  2090 & 332 & 148 & 26 & 1 \\ \hline
        1m sea level rise & 127 & 64 & 9 & 0 \\ \hline
        2m sea level rise & 343 & 155 & 29 & 1 \\ \hline
        3m sea level rise & 584 & 285 & 55 & 3 \\ \hline
        4m sea level rise & 752 & 335 & 70 & 5 \\ \hline
        5m sea level rise & 1023 & 402 & 77 & 8 \\ \hline
    \end{tabular}
    \caption{Impact of Sea Level Extremes - Buildings Flooded \cite{kartverket_se_2021} The major contrast between current 20 year storm surge and the 2090 20 year storm surge is that at this level a critical building is flooded. This would also occur at 2m sea level rise.}
    \label{building-impact-sle}
\end{table}


Table 4.1 shows that there is some risk from sea level extremes now. The 20 year return height is projected to cause the flooding of 247 buildings. However it is worth noting that no critical buildings are predicted to flood even with the 1000 year return height. This is in contrast to the 20 year return height in 2090, which is projected to cause the flooding of one critical building.

Norway's economy and residency patterns are very dependent on activities along the coast\cite{aunan_strong_2008}, as can be seen above Trondheim also has some dependency on coastal activities. The information here will be used to discuss Trondheim's technological resilience to sea level extremes in the discussion. 


\section{Natural Systems}
maps - old version used table i think a map would be better



\section{Social Systems}



\subsection{Summary Statistics of Survey Results}

%don't want to discuss factors pr variable names - how to do this best without that 

\subsection{Place and language}

\begin{table}[h]
    \centering
    \begin{tabular}{|l|l|}
    \hline
    Place  & No. Subjects  \\ \hline
      Skansen   & 29    \\ \hline
      Nidelva & 57      \\ \hline
      Grillstad & 37       \\ \hline
      Brattøra & 30     \\ \hline
      Total & 153   \\ \hline
     \end{tabular}
    \caption{Place subjects chose to Answer on. The most popular place to Respond was Nidelva with 57 response representing 37 percent of responses. Next popular was Grillstad with 37 responses and then Brattøra with 30 and Skansen with 29 respectively. The reasonable spread of response does allow for comparison}
    \label{tab:place}
\end{table}
\paragraph{}

As can be seen in the table above the most popular place for responses was Nidelva. This is the most central of the locations and has  the greatest daily throughput of people. It also includes perhaps the most iconic views of Trondheim. Next popular was Grillstad, perhaps due to the recognition by residents that the area could be severly influenced by flooding from sea level extremes. Skansen and Brattøra are the next most responded to, both of these locations have significant commercial ventures. Brattøra in particular is dominated not by residency but by offices and industry. Perhaps the conduction of this survey in Summer decreased the number of responses due to the lack of office workers. 
Almost evenly split for each location was whether the survey was completed in Norwegian or English. English surveys had 66 response, while the Norwegian survey had 87 responses.  

\subsection{Interest Level in Sea Level Extremes}

\begin{figure}[h]
    \centering
    \includegraphics{fig_results/interest-level.png}
    \caption{Interest Level in Sea Level Extremes}
    \label{fig:my_label}
\end{figure}
\paragraph{}
words words words


\subsection{Memory of Sea Level Extremes}

\begin{figure}[h]
    \centering
    \includegraphics{fig_results/memory-sle.png}
    \caption{Memory of Sea Level Extremes}
    \label{fig:my_label}
\end{figure}
\paragraph{}
words words words
\subsection{Level of Interest in Sea Awareness}

\begin{figure}[h]
    \centering
    \includegraphics{fig_results/2022-20yrss-answer.png}
    \caption{Level of Interest in Sea Level Extremes}
    \label{fig:my_label}
\end{figure}
\paragraph{}
words word words



\subsection{Information Access about Climate}

\subsection{Community Membership}

\begin{figure}[h]
    \centering
    \includegraphics{fig_results/com-mem-horizontal.png}
    \caption{Membership of Communities}
    \label{fig:my_label}
\end{figure}
\paragraph{}
words words words 
\subsection{Access to Survey}

\begin{figure}[h]
    \centering
    \includegraphics{fig_results/access_survey.png}
    \caption{Access to Survey}
    \label{fig:my_label}
\end{figure}
\paragraph{}

\subsection{Perceived Risks}

\begin{figure}[h]
    \centering
    \includegraphics{fig_results/infrastructure-risks.png}
    \caption{Infrastructure Risks}
    \label{fig:my_label}
\end{figure}
\paragraph{}
words words words

\begin{figure}[h]
    \centering
    \includegraphics{fig_results/people-risks.png}
    \caption{People risks}
    \label{fig:my_label}
\end{figure}
\paragraph{}

\section{Awareness}

\begin{figure}[h]
    \centering
    \includegraphics{fig_results/2022-hightide-answers.png}
    \caption{Subjects Predicted Height of Current High Tide. Height 0.5m is the neep high tide, Height 1.2m is the Spring High Tide. Th vast majority of subjects got an answer which corresponds with models of tides in Trondheim. Under 20 subjects responded with an answer which does not correspond with models of tides. 58 subjects chose 0.5m, 77 subjects chose 1.2m meaning that 135 out of 153 subjects, 88 percent chose what could be considered the correct answer for high tide. While only 13 subjects chose 2.2m and 3 subjects chose 2.7m}
    \label{fig:high-tide-answer}
\end{figure}
\paragraph{}
As can be seen in figure ** above the subjects displayed a high awareness of the tides in Trondheim. Under 20 subjects responded with an answer which does not reflect the models from \cite{kartverket_se_2021}. Over 88 percent of subjects chose a value which matches with models by \cite{kartverket_se_2021} The majority of respondents chose the answer of 1.2m which corresponds with the Spring tide. 

\begin{figure}[h]
    \centering
    \includegraphics{fig_results/2022-20yrss-answer.png}
    \caption{Subjects Predicted Height of Current 20 Year Storm Surge. 61 subjects chose 1.2m this is very comparable to the 60 which chose 2.2m which is the answer which corresponds with \cite{kartverket_se_2021}. 24 subjects chose 2.7m and only 6 subjectts chose 3.0m}
    \label{fig:2022-stormsurge-answers}
\end{figure}
\paragraph{}
The majority of subjects chose 1.2m as the predicted height of the 20 year storm surge, this does not correspond with models from \cite{kartverket_se_2021}. In fact 1.2m is equal to the current high tide. Almost equally well chosen was the answer which does correspond with models from \cite{kartverket_se_2021}, 2.2m, with 60 subjects choosing this answer. This means that 43 percent of subjects chose the answer which corresponds with \cite{kartverket_se_2021}.

\begin{figure}[h]
    \centering
    \includegraphics{fig_results/2090s 20yr ss answers.png}
    \caption{Subjects Predicted Height of 2090's, 20 year storm surge. The majority of subjects chose the highest value with 70 subjects choosing 3.0m as the predicted height of the storm surge. 47 subjects chose 2.7m which is the value which corresponds with models by \cite{kartverket_se_2021}. 27 subjects chose 2.2m and only 5 subjects chose 1.2m}
    \label{fig:2090-stormsurge-answers}
\end{figure}
\paragraph{}
The majority of subjects predicted the storm surge in 2090 to be 3.0m. Just over 30 percent chose 2.7m which is the value which corresponds with models from \cite{kartverket_se_2021}. This means that most people thought the sea level extreme associated with the 20 year storm surge in 2090 is higher than it is currently predicted to be. The change of height for the 20 year storm surge predicted for 2090 is only 50cm higher than the current 20 year storm surge \cite{kartverket_se_2021}. 
\paragraph{}

\begin{figure}[h]
    \centering
    \includegraphics{fig_results/slr-future.png}
    \caption{Subjects Predictions for how Sea Level will change in Trondheim in Next 30 years. Only 147 subjects answered for this question, unlike every other one which all subjects, 153, answered. No subjects believed the sea level will decrease over the next 30 years. One subject answered that it will stay the same, with every other subject chosing that it will rise by some level }
    \label{fig:my_label}
\end{figure}

There is strong uncertainty with this measurement in models, due to the unknown of emisson patterns over the next 30 years and isostatic uplift. Never the less an estimation of 20cm or 10cm is in line with \cite{kartverket_se_2021} which uses the upper emissions pathways from the IPCC. Over half of subjects chose an answer which is in line with models from \cite{kartverket_se_2021}. However equal numbers of subjects,26, chose 50cm which is not in line with these models as chose 20cm. Almost all subjects answered that sea level will rise in the next 30 years, though six subjects chose not to answer this question.  

\begin{figure}[h]
    \centering
    \includegraphics{fig_results/slr-past.png}
    \caption{Subjects Predictions for how Sea Level has Changed in Trondheim over last 30 years. Only 3 subjects chose that sea level has decreased in the last 30 years, this answer is inline with \cite{kartverket_se_2021}. 20 subjects answered that sea level had not changed over this time. While 31 subjects answered that it had risen by 10 cm. 18 subjects answered that it had risen by 20 cm and 3 answered that it had risen by 50 cm. 52 subjects answered that sea level had risen over the last 30 years. }
    \label{fig:my_label}
\end{figure}

The next figure also 
\begin{figure}[h]
    \centering
    \includegraphics{fig_results/Aware_sea_level_change.png}
    \caption{Awareness of Sea Level Change. Only 3 subjects chose an answer which corresponds with the models from \cite{kartverket_se_2021}, for what height sea level in Trondheim will be in 20 years time. In contrast 82 subjects just over half chose an answer which corresponds with models from \cite{kartverket_se_2021} for what the sea level was 30 years ago. }
    \label{fig:my_label}
\end{figure}
\paragraph{}

For both of these questions 

Figure.... below is the sum of all results pictured above in this section on awareness. This was the first attempt to determine awareness of subjects from the results of this survey

\begin{figure}[h]
    \centering
    \includegraphics{fig_results/aware_all.png}
    \caption{Awareness Considering all Questions which were designed to determine awareness}
    \label{fig:aware-all}
\end{figure}
\paragraph{}

Not very aware - want better split - new determination of awareness -> exclude awareness of sea level change from it .

\begin{figure}[h]
    \centering
    \includegraphics{fig_results/aware_all.png}
    \caption{Awareness considering all Simulated Water Level Questions}
    \label{fig:aware_all}
\end{figure}

Due to the reasons expanded upon in the discussion of results, awareness determination considering all answers related to questions was not the most suitable for analysing factors related to awareness. For this reason another determination of awareness was created utilising only the questions which had pictures of the simulated water levels. T

\paragraph{}
As shown by the focus group there is issues with just requesting a numeric value from subjects. There is a higher potential connection for many individuals including another emotional layer when using pictures rather than numbers. 

\section{Factors Affecting Awareness}

\subsection{ Awareness and Place}

\subsection{Awareness and Language}

\subsection{Awareness and Community Membership}

\subsection{Awareness and Level of Interest}

\subsection{Awareness and Information Access}