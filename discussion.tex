% LIMITS OF SURVEYING AS TECHNIQUE
% COMMUNICATION RISK - REACHING OUT TO SUBJECTS


%size of city plus no major disaster impacts results and discussion
%seasonal aspect

%remember survey was difficult - can say quite aware even if only got 2 correct





\chapter{Discussion of Results}

\section{Technological Systems Resilience}

\section{Natural Systems Resilience}

\section{Social Systems Resilience}

\subsection{Place and Language}

%issue comparing norwegian and english surveys - word just dont mean the same - direct translation is impossible -cultural differences when viewing risk

\subsection{Interest Level in SLEs}
Interest level in SLEs does not appear to be a factor influencing Stakeholders awareness about SLEs in Trondheim. 

Interest level had no correlation with the physical vulnearbility of the place the subject chose to respond on either. 

\subsection{Memory of SLEs and Length of Place-based Knowledge}
\subsection{Information Access about Climate and Place}
\subsection{Community Membership}
\subsection{Access to Survey}
\subsection{Perceived Risk}

\subsection{Risk Perception - Personal Impacts of SLEs and associated Flooding}
There was opportunity for subjects to give extra information on certain topics as can be seen in the appendix. From these responses narratives about perception of risk can be explored. For example their was a very varied response to "How would flooding associated with SLEs in this area affect you? If you would like to give more details on this, please write below." This has been grouped into narratives depending on the impact level the subjects gave to the previous question.
\paragraph{}

Subjects who felt they would have no impact who gave more details either stated that they lived far from the sea, or that they would be moving away soon. This highlights an idea that if the sea level extreme does not impact items they own or their daily activity then it will not affect them at all. Nuisance flooding was not considered an issue, nor how flooding could impact key infrastructure for example transport connections. One subject gave details of how flooding did impact where they normally park their car, but as it wasn't parked during this event then they weren't impacted. 
\paragraph{}

Subjects who responded that flooding in this area would have mild impact and gave more information gave a wider range of responses than those who felt it would not impact them. The overall narrative was that flooding won’t affect where they live, but may impact their family and their activities. One curious response was from an individual who said that because they lived on a higher floor they would not be impacted. Living on a higher floor may provide some protection but in a extremely high sea level could still have impacts on daily activities. This continues the narrative that SLEs may impact my activities but as long as it doesn't negatively impact my possessions the impact can be considered mild. This raises questions about what it means to experience a flooding hazard. What it means to have direct experience can have varied definitions. 
\cite{whitmarsh_are_2008} defines direct flooding experience as those who have experienced forms of flood damage to their home, garden or vehicle within the last half decade. This is a particularly tight definition of direct experience and focuses on ownership. People who do not own a home, garden or vehicle can still be impacted by flooding and their experience may be very different.  
\paragraph{}
It is commonly assumed that direct experience makes people more concerned about climate change and its potential personal impacts, but this does not always appear to be the case (\cite{lujala_role_2020}). Perhaps some of this variation is due to variation in what is defined as direct experience. Further research could look into differing types of direct experience. 
\paragraph{}
Only once subjects responded with medium impact did a negative emotional response get expressed. Subjects discussed how flooding will affect places where they stored things, particularly at the harbour. Subjects who ranked flooding in this area as Significant impact gave similar response. The overall narrative expressed from the extra information given here is that subjects appear to believe that if SLEs do not affect the metres of where they live it will not affect them. 
\paragraph{}
None of the subjects who gave extra information mentioned insurance as a factor in their perception of personal impact level of flooding due to SLEs. Gonzalez-Riancho et al (2017 as cited in \cite{gerkensmeier_governing_2018}) says that stakeholders for the Wadden Sea Region a similarly developed area in Northern Europe, who has a longer history of negative flooding impacts from SLEs, reject that insurance can be considered as a form of disaster risk management. Perhaps the self-selecting stakeholders for Trondheim have similar views. How economists view the role of insurance in the case of resilience to flooding often does not correspond with other stakeholders, particularly residents (\cite{gerkensmeier_governing_2018}).

%Experience types (of flooding) - damaged belongings, survived, nuisance, direct observation, family/friend observation, heard about e.g. media


There is widespread insurance of buildings in Norway against damage from natural hazards (\cite{lujala_role_2020}). Households with fire insurance are required to also have insurance for the damage caused by landslides, storms and flooding. The cost of this is connected to the insurance rate for fire damage and is not dependent on the location of the house (\cite{lujala_role_2020}). The role of insurance and the welfare state in the perception of risk in Trondheim is likely significant (\cite{lujala_role_2020}), but this was not mentioned by any of the subjects regardless of their perception of the potential personal impact of flooding in these places. Future research could include questions to gain a greater understanding of how stakeholders view the role of the welfare state or insurance policies in their perception of this risk. 

\subsection{Risk Perception - Beyond Sea Level Extremes}
Subjects were given the opportunity to highlight other risks to people and infrastructure which were not mentioned by the researcher. They highlighted risks associated with the natural systems and the interplay between the technological and social system. The concern about the risk of land erosion is displayed in the quote below from a subject who ranked the impact of flooding the area as significant.

\begin{itemize}
    \item "A few days ago, I saw a video of some part if Norway where an entire chunk of land just went under the water, I guess due to erosion, and I would assume that storm surges have an effect on erosion. That video was horrifying and I keep thinking, if something like that to happen here in Trondheim, which part of coast would it take away. I would love to buy/rent a house near the coast, but after watching that, I am not so sure."
\end{itemize}
\paragraph{}

At a surface level this quote highlights a fear of flooding and that they like other subjects who labelled their impact as no impact will avoid living near the coast in part due to fear of flooding. But taking it further the fear here isn't simply of SLEs but of land falling away. They are concerned about the land erosion caused by SLEs.
\paragraph{}
This fear of land falling away is brought up by several subjects at multiple points when given the opportunity to broaden the survey. For example when asked "If you would like to give other examples of risk in this area, please write here." by far the most common response was risk of landslide - often specifically due to quick clay. This response was mainly given by those taking the survey in Norwegian. 
\paragraph{}
Stakeholders awareness of SLEs appears dependant on information sources about the changing climate. Awareness appears dependant on the the stakeholders use of newspapers, family and formal education. Quick clay and landslides get media attention in Norway and this may impact awareness and resilience for these natural hazards. This could be explored in a conduction of multi-risk resilience for Trondheim.  
\paragraph{}
The other responses were about risks associated with the interplay between technological and social systems. The most common of these responses focused around lack of housing. This was sometimes discussed in the context of people having to move due to changing climate, including rising SLEs, but not always. This view is the flip side of the many respondents who stated that they will minimise personal impact of flooding due to SLEs by not residing along the coast.  As stated in the background, Trondheim has 34.6 percent of the land use in its coastal area influenced by buildings (\cite{engebakken_construction_2022}), a significant percentage which are residences. The impact to housing in Trondheim is likely not that significant within the next 70 years, but the impact in the desire to live there may be greater. This push inland could have significant economic, social and ecological impacts in Norway. 
\paragraph{}





\subsection{Determination of Subjects Awareness}
lack of marine workers... 

\subsection{Factors Affecting Awareness}
lack of factors....

\subsection{Limitations of Data Analysis Techniques }



\section{Research Question 1 - How resilient will Trondheim to be to SLEs during 2022 to 2050 and 2050 to 2100? }
%natural systems resilience
%technological systems resilience
%social systems resilience
From the literature review Trondheim's natural systems of resilience and technological systems of resilience can be considered very high for the period of 2022 to 2050. From the results of the survey the  social systems of resilience for this period are only considered as medium as the awareness determined for the current SLEs are only somewhat. Overall Trondheim's resilience to SLEs can be considered high, due to the limited damage potential and the high level of resources available to deal with it. However this does not mean that there is not potential economic, cultural or social impacts. Just that it is projected that the daily activities occurring in Trondheim will quickly return to normality after the likely sea level extreme events during this period.
\paragraph{}
The awareness determined for SLEs in 2090 is higher than the awareness for 2022. This allows for the projected social system resilience for the period of 2050 to 2100 to be high. The technological system resilience is also considered as high for this period and the natural system resilience is considered medium. Overall the projected resilience for Trondheim to SLEs during 2050 to 2100 is high. However, projected resilience is highly dynamic, meaning that the resilience of a place will change over time. There is requirements from the Sendai framework for and the United Nations sustainable developments goals (UN SDGs) 11 and 13 that cities resilience to natural hazards should increase (\cite{gonzalez-riancho_storm_2017}). To see if this is occurring repeated measurements of resilience is required. 

This determination of current resilience levels is backed up by \cite{opach_seeking_2020}. Who conducted a cluster analysis of community resilience for Norway and ranked Trondheim as having high economic, housing and infrastructure resilience. They were designated with lower institutional and community capital decreasing the social system resilience. However it can be argued that the way these valuations for resilience were determined were biased against cities as and favoured less populous areas.

\paragraph{}
This requirement of increasing resilience brings up the question of what level of resilience should a place have? What level is acceptable to stakeholders is suggested as a method of answering this question by \cite{gerkensmeier_governing_2018}. The four research sites of Skansen, Grillstad, Brattora and Nidelva represent the most vulnerable places in Trondheim to SLEs. Taking the extreme potentials of SLEs, helps minimise the chance of overestimating the resilience Trondheim has to SLEs. The results from the survey indicate that the place had no impact on how aware subjects were to SLEs, even though each of these places had different level of natural and technological system resilience. 




\section{Research Question 2 - Are Stakeholders aware about changes to SLEs?}
%people are somewhat aware
Stakeholders are somewhat aware about changes to SLEs. The results displayed in Figures 5.24
show that when displayed the information about SLEs using edited photographs and numeric valuations, 13 percent of subjects got all answers correct. 40 percent of subjects got one answer correct. Almost 40 percent of subjects got two out of the three answers correct. Only 7 percent of subject got none of the answers correct. 

 Stakeholders do not appear aware of the changing sea level extreme over the past 30 years. Very few responses highlighted that sea levels in Trondheim have decreased over the last decades. This rises questions about memory. The level of difference that is required for individuals to notice and remember is particularly important here. There has been only *** cm change in sea levels in Trondheim, was this simply too small for people to notice. 

\paragraph{}

\paragraph{}
Stakeholders awareness could be increased, hence improving social system resilience. The factors which impact stakeholders awareness and whether they can be used to increase overall resilience are discussed in the next section 

\section{Research Question 3 - What factors impact stakeholders' awareness of SLEs? }
%more difficult as not so many aware
%could try and reverse it - what makes people unaware 
%interest level doesn't appear to be a factor in awareness which contrast with 


The choice which format method to use while asking about sea level changes is likely to have an influence on how participants perceive this change. For example considering change as metres in height vs area of land influenced may results in different perceptions of the changing risks and potential impacts.  By sticking firmly in the realm of historical fact and scientific models with these questions there has been an attempt to investigate awareness of SLEs rather than perception. 

\section{Visualisation of sea level change}
Discuss exclusion of questions "How much do you think the sea level will change in the next 30 years?" and "How much do you think the sea level has changed here in the past 30 years?". 

Especially why focus on answers which were picture based versus answers that were number based


%EDUCATION AS AN OPTION WAS ONLY ADDED AFTER day 1
%as was peer reviewed articles for the norwegian options
%so first 20 respondents did not have that option














\section{bits cut from results }
Trondheim city plans indicate that the research sites should be prepared for 4.87m by 2100.



Table 5.2 shows that the most popular place for responses was Nidelva.Nidelva is the most central of the locations and has the greatest daily throughput of people. It also includes perhaps the most iconic views of Trondheim.
  Next popular was Grillstad, perhaps due to the recognition by residents that the area could be severely influenced by flooding from SLEs. Skansen and Brattøra are the next most responded to, both of these locations have significant commercial ventures. Brattøra in particular is dominated not by residency but by offices and industry. Perhaps the conduction of this survey in Summer decreased the number of responses due to the lack of office workers. Almost evenly split for each location was whether the survey was completed in Norwegian or English. English surveys had 66 response, while the Norwegian survey had 87 responses.  


  Awareness as a facet of local knowledge was the key variable within this project. The purpose of the survey was to allow for the determination of awareness in Trondheim of SLEs. An interesting comparison is the level of interest against awareness. There is the assumption that high or professional interest in SLEs is associated with higher levels of awareness of the risk. 


  There is no linearity or other pattern observed in Figure 5.11. This raises questions about the assumption of resilience due to the presence of professionals within a community. However, there is a serious limitation: no marine workers were identified in the survey subjects. Subjects who ranked their interest in sea level as high (4) had the largest variation in determined awareness according to the number of answers they gave which matched models from \cite{kartverket_se_2020}. 


  How memory is formed and how it impacts awareness and resilience is a highly debated field \cite{de_guttry_expiry_2022}.

  The lack of memory is interesting considering that length of knowledge for 95 out of 153 subjects the majority included at least one sea level extreme events as can be seen in the figure below. 


  While access to survey does not appear to have influence on awareness of local knowledge, the results are important for the discussion on how to create a framework for determining resilience of a place.

   However almost all subjects only chose one access type as seen 5.15

The key result for future research is that data collection using posters was a popular access method. 50 posters were printed. To see the communication design principles influence on the poster design, you can find it attached in the appendix. These posters were split evenly between the research sites and half were placed at the start of data collection and the other half after the first month. The results shown in figure 5.16 are again influenced by the subjects tendency to tick only one box, when there was the option to tick several. For example, it is known that many of the subjects who have personal connections to the researcher actually accessed the survey via social media or organisational membership. This tendency and its influence on the results is expanded upon in the discussion of results and the discussion of framework.



25 subjects responded that they accessed the survey via personal connection to the researcher. The effect of the researchers pre-existing local knowledge and pre-existing network on this research is not expanded upon here. Though there may have been some impact, this result indicates that if this project was done in a location where the researcher does not have a connection would still yield results.

As can be seen in the appendix survey questions about the perceiving of risks were only asked about towards the end. The information received and subjects mentioned earlier likely influenced the results of these questions as can be seen in figure 5.17 and 5.18 .


This survey will have influenced the subjects perception of risk, it is unlikely that 97 subjects would have selected storm surges as a risk to infrastructure before carrying out this survey. The impacts of citizen science and perception of risk is expanded upon in the discussion of framework, including the high levels of perception that shoreline instability is a risk to infrastructure in the research sites.



The determination of awareness displayed in Figure 5.25 was chosen as the most appropriate variable due to the fact that only awareness questions with a simulated water level picture were included, as can be seen in the appendix. The other determination of awareness included the results from two questions which were based purely off numeric answers which gave very skewed results. When given just numeric responses, the accurate rate was non significant. For example, only 3 subjects out of 153 gave the correct answer for projections of sea level change in the last 30 years as shown in Figure 5.23.  This is unlikely given that it was multiple choice with only 7 potential answers (if split evenly then would have 21 subjects choosing this response. Figure 5.21 subjects predictions for how sea level will change in the next 30 years has more correct responses. This was only mildly significant, which combined with the lack of correct answers as shown in figure 5.22 was considered a valid reason to exclude the answers to both of these questions for creating the variable of awareness. 


Another determination of awareness was considered which only includes results from questions on current storm surge and current high tide, excluding results of the question of the 20 year storm surge in 2090. This was considered as a potential baseline variable for current awareness to SLEs as different from awareness to future SLEs. However, the variable summary statistics are similar.



The results from the survey for both the variables and factors were skewed as can be seen from the figures in the section above. The analysis was carried out using RStudio and then exported to Microsoft Excel to utilise their templates for graphs. The results from the histogram analysis is that no variable of awareness appeared normally distributed. Upon basic transformation, including logarithmic and cubing, it was highly skewed. Furthermore, almost all factors from the results were skewed as can be seen in the section above. 


SUMMARY OF RESULTS...
The key result from the pilot survey and focus group is that determining awareness from surveys was an appropriate method for use in determination of social resilience to SLEs in a place. Also, that utilisation of maps was not appropriate for the research sites chosen due to the low level of projected changes and that simulated images of visual extremes were a better choice. From the data-sets used and literature review, technological and natural systems projected changes in vulnerability and how they impact resilience were outlined. With current city organisation, only once SLEs reach the height of 269cm, which is the projection for the 20 year return height in 2090, is a critical building due to be impacted. Information access and community membership appear to be factors in the determined awareness of subjects. Over half of subjects select storm surges as a perceived risk to people and infrastructure. Subjects appear to be somewhat aware of the changing SLEs. The format in which questions of SLEs are asked appears to have an impact. 


\cite{cutter_community_2020} discusses difference between insurance and resilience - how economic resilience can be impacted at lower levels of flooding impact