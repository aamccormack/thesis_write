

\chapter{Discussion of Results}
The results of the previous chapter are evaluated here and then these findings are put in the context of the theory. After this, the findings are used to answer the three research questions. 



\section{Technological Systems Resilience}\label{tech-resilience-discussion}
The three systems of resilience based on \cite{cutter_place-based_2008} are highly interactive. Distinguishing technological systems from natural systems for the risk associated with SLEs is particularly difficult in Trondheim, due to the long history of the coastline being shaped by its population. Table \ref{table:research_site} in Chapter 2 - Background, displays how each of the research sites' coastlines have been artificially altered. The dominant use of the space, the land type, and presence of protection, such as harbour walls, dictate the assigned physical vulnerability (\cite{opach_seeking_2020}).
\paragraph{}

With the current city layout, only once SLEs reach the height of 269cm, which is the projection for the 20 year return height in 2090, is a critical building due to be impacted (\cite{kartverket_se_2021}). This does not mean that there is no impact to infrastructure, but that it is likely that the technological systems in Trondheim are able to quickly return to normality for SLEs below this level. However, Trondheim Municipality's plans indicate that the research sites should be prepared for an SLE  of 4.87m by 2100 (\cite{hanssen_saksframlegg_2013}).
\paragraph{}
Once an SLE of 4m is reached, 1,162 buildings will be flooded, 5 of these being critical buildings (table \ref{table:building-impact-sle}). If the current organisation of Trondheim remains, this is projected to occur in the period 2080 to 2100. 2080 is 58 years away, meaning there is a good chance of the city layout shifting and, even if it remains the same, there is not a high likelihood of this level of flooding impacting the city.  
\paragraph{}
The ability to return to normality is only possible for sites which are temporally flooded. By 2090, 60 buildings in Trondheim are projected to be permanently flooded. These buildings are not considered as possessing resilience, but they are a small part of the overall technological system. The impact of permanent flooding and nuisance flooding and how it interacts with Norway's system of insurance is discussed in the section on Risk Perception. While these floods may have minimal effects on the technological system resilience, they are likely to affect the economics of the sites, including increases to insurance premiums (\cite{cutter_community_2020}). 
\paragraph{}
Residency patterns are mainly considered under Social Systems of Resilience. However, the size, spread and density of buildings impacts resilience to natural hazards. In the case of resilience to SLEs, this is particularly the case if - as has been suggested - there is more intrastate and residency moved onto reclaimed land such as seen in Grillstad. If this occurred, it could decrease Trondheim's technological system resilience.
\paragraph{}
While some infrastructure in Trondheim is projected to be resilient to future SLEs (for example, ...), the drainage network and tunnels are at greater risk (\cite{hanssen_saksframlegg_2013}). This is due to their lower, often subterranean position. The drainage network is perhaps the greatest weakness in the technological system's resilience, but there is ongoing work to improve it (\cite{hanssen_saksframlegg_2013}). This again highlights the need for repeated measurement to understand a place's trend in resilience. While vulnerability is here considered as a value, resilience is an ongoing, dynamic process.
\paragraph{}
Whilst we can measure technological system resilience and rank it - low, medium or high - this does not include the stakeholders' judgement of an acceptable resilience ranking. What level of resilience is deemed acceptable by stakeholders could be an interesting subject for future research and likely of particular interest to policy makers.
\paragraph{}
Overall, the technological resilience of Trondheim for the period 2022 to 2050 is considered very high; 2050 to 2100 is considered high. Even with the increased risk, the building regulations implemented in 2016 are preparing for this risk. These regulations are not designed to fully prevent flooding, but rather to minimise impact and enable a quick return to normality. 


\section{Natural Systems Resilience}
The projections of SLEs in this thesis were based on the fifth assessment of IPCC RCP 8.5 emission scenario: this is considered the likely emission projection for business as usual (\cite{hanssen-bauer_climate_2017}). This scenario was chosen due to the uncertainty of calculation of SLEs, particularly those associated with storm surges. By potentially overestimating the impact of SLEs, it is hoped that this prevents overestimating the resilience of places to SLEs. Furthermore, it is highly possible that emission patterns will match those of RCP 8.5. 
\paragraph{}

For Trondheim, it is projected that a sea level change of -1 to + 21 cm will occur between 2022 and 2050, then a further sea level rise of 22 to 77 cm between 2050 and 2100 (\cite{hanssen_saksframlegg_2013}). The most important factors when considering Trondheim's natural systems resilience are storm surges. Storm surges are expected to be problematic due to the so-called accumulation effect related to the shape of Trondheimsfjord, which is long and deep  (figure \ref{fig:research_area}). While Trondheimsfjord currently gives Trondheim city protection against storms washing in from the North Sea, the accumulation effect may exaggerate the impacts of storm surges (\cite{hanssen_saksframlegg_2013}). 
\paragraph{}
Modelling of relative sea-level change has significantly improved over the last decades, moving from a bathtub model to including impacts of isostatic uplift. A much improved understanding and monitoring of glacioisostasy (i.e. vertical land-level changes due to loading and unloading of the Earth's surface during glacial-interglacial cycles) is particularly important for projecting future RSL change in Norway. The results from the survey in this study indicate that the subjects have a lack of awareness of the impacts of glacioisostatic uplift in Trondheim. Relative sea level in Trondheim has actually decreased over the last 30 years, mainly due to glacioisostaic uplift. There is a slight uncertainty in this value (15cm +-5cm; reference?), but relative sea level has certainly fallen during this period of time. That only three subjects correctly identified this trend (figure \ref{fig:slr_past}) may highlight how people view their location as static.

\paragraph{}
Surprisingly, subjects predicted the future better than the past when asked about sea level changes in Trondheim (figure \ref{fig:slr_past} and figure \ref{fig:slr_future}). These patterns of awareness level may be explained by consulting narratives of SLEs in the media, where there is a focus on rising sea levels. Hence, the subjects expected that sea levels have already risen in Trondheim, contrary to the recorded evidence from \cite{tides_high_2022}.


\paragraph{}
An alternative explanation would be that they are questioning the evidence of falling past sea levels. There is a long history of undervaluing local knowledge in the physical sciences (reference?). However, trends in relative sea level are particularly difficult to observe in Trondheim due to a large tidal range and frequent weather effects (waves, storm surges). I suggest that local undervaluing of local scientific knowledge is a stronger argument for the differences between the models and subjects' responses. Determining sea level change requires repeated measurements and the calculation of the impact of tide and weather. This process needs to occur for many years and the measured change is rather small and would be difficult for individuals to spot. The tide gauge in Trondheim was moved in 1990 and it was for this reason that 30 years was selected as the time frame for these numeric questions. 

\paragraph{}
Subjects in this study had much higher "correct" answer rates for the questions which utilised edited photographs of simulated SLEs, rather than purely numeric. This indicates that edited photographs are a better method for discerning stakeholders' awareness when it comes to relative sea-level change. The numeric questions had seven choices, while the edited photograph questions had only four. This disparity may also have had an impact, as the statistical likelihood of choosing the "correct" answer increases with fewer options. When it comes to  observing SLEs, a photograph is closer to the lived experience than a numeric value. Numeric values were included in these questions in response to outcomes from the focus group. It was suggested that some individuals may have a direct knowledge of numerical values. Also it allows those with poor eyesight to fully participate, even if using an e-reader.
\paragraph{}
Natural Systems Resilience to SLEs in Trondheim is considered here as high for both 2022 to 2050 and 2050 to 2100. The flood levels projected from SLEs will not prevent a quick return to normality. However, the changing climate and ecological systems when taken as whole are much less resilient. The appropriateness or otherwise of using single-hazard analysis of resilience is outlined in the section Discussion of Framework.



\section{Social Systems Resilience}
The focus of Social Systems Resilience thus far has been on local knowledge, awareness and community resilience. There are other facets of social system resilience including economic and cultural resilience. Cultural resilience is further discussed in the section on Perceived Risk. Although economic systems are vulnerable to disasters (\cite{head_comment_2020}), Norway and Trondheim have strong economic systems at present, with large amounts of buffer. However, repeated disasters will decrease this resilience. Of particular interest for SLEs, is how insurance and resilience interact.

As mentioned earlier, there is evidence that many stakeholders (especially residents) do not consider having insurance as being connected to resilience. How insurance affects individuals who have experienced damaged or lost property is discussed in \cite{whitmarsh_are_2008}. It has also been highlighted that economic resilience can be impacted at lower levels of flooding than other social systems (\cite{cutter_community_2020}). Currently in Norway, household flooding insurance is not dependent on location (ref).  How insurance will deal with increasing risks from natural hazards caused by the changing climate is of concern in Norway, as it is worldwide. The other aspects of Social System Resilience are explored in the subsections below.


\subsection{Place and Language}
Table \ref{tab:place_language} shows that the most popular place for responses was Nidelva. Nidelva is the most central of the locations and has the greatest daily throughput of people. It also includes, perhaps, the most iconic views of Trondheim. Due to the prioritisation of video on Facebook and Instagram, a short video displaying the changing SLEs projected for Nidelva was attached to each social media post. This may be part of the reason that Nidelva was the most popular place for subjects to respond on.  Next popular was Grillstad, perhaps due to the recognition by residents that the area could be severely influenced by flooding from SLEs. The researcher has few personal connections to this place, so this high level of response was particularly interesting. Grillstad is situated on recently reclaimed land (figure \ref{fig:research_site}), which may have influenced the response level. Skansen and Brattøra are the next most responded to; both of these locations have significant commercial ventures. Brattøra in particular is dominated by offices and industry. Perhaps conducting this survey in summer decreased the number of responses due to the lack of office workers. 
\paragraph{}
English surveys had 66 responses, while the Norwegian survey had 87 responses. There is no such thing as a perfect direct translation. The nuance of the terms can be different and even when the words have the same meaning there are cultural differences in the conceptualisation of these terms. Also, there may be cultural differences between the subjects in their conceptualisation of risk and resilience. Furthermore, the increased likelihood of subjects responding in English having lived in another location first may influence their view of Trondheim's resilience. How language used impacts an individual's view of risk and resilience is being researched by ***.

\paragraph{}
It is worth noting that due to the set up of the survey, the migration status of the subjects is unknown, as is their first language. English and Norwegian were chosen as the survey languages due to the number of people who can communicate in either of these languages. If this thesis was to be repeated in a different location, survey languages used should be carefully considered.


\subsection{Interest Level in SLEs}
 An interesting comparison from the survey results is the level of interest against awareness. There is evidence that community interest in natural hazards improves awareness and resilience (\cite{cutter_community_2020}). Interest levels in SLEs does not appear to be a factor influencing stakeholders' awareness about SLEs in Trondheim. There is no linearity or other pattern observed in figure \ref{fig:aware_vs_interest}, which raises questions about the assumption of resilience due to the presence of professionals within a community. However, no marine workers were identified in the survey subjects and it could be easily argued that they have an interest in SLEs. Subjects who ranked their interest in sea level as high (4) had the largest variation in determined awareness according to the number of answers they gave, which matched models from \cite{kartverket_se_2021}. Thus in this context, general awareness of a subject does not necessarily translate to local knowledge.
\paragraph{}
Furthermore, interest level in sea-level change? had no correlation with the physical vulnerability of the chosen place. It was hypothesised that interest level would be dependent on this factor. The lack of subjects who identified as having professional interest (12) could be a limiting factor here. The tendency for individuals to select the middle answer is likely part of the reason why almost half the subjects selected that they have a medium interest in SLEs. One proposal is that those with higher or above levels of interest in SLEs simply did not have knowledge of the specific places they answered on. Trondheim has not had recent disasters associated with SLEs. How well high levels of academic knowledge relate to local knowledge is an ongoing aspect of study for resilience to natural hazards (\cite{lujala_role_2020}).
\paragraph{}



   

\subsection{Memory of SLEs and Length of Place-based Knowledge}
As shown in figure \ref{fig:memory_sle}, data on memory of SLEs in Trondheim is skewed. Memory connects to length of knowledge and the length of knowledge for the majority of subjects does not extend far enough back in time (see figure \ref{fig:long_know}) to include all of the major events in Trondheim in the survey. It was highlighted in the results that 96 subjects had a length of knowledge of SLEs in Trondheim greater than five years. This time period includes the SLE event occurring in February 2020, for which only 51 subjects responded that they remembered. There are many potential reasons why 45 subjects who had a long enough length of knowledge did not remember this event. First, they were unaware of the event occurring, perhaps due to lack of personal impact, the short lived nature of the event, or not being present in Trondheim at the time. Secondly, peoples' understanding of timelines can be limited: even if they remember the event the date may not be well associated. This is part of the reason why specific dates were not mentioned in this question, just years and months. The media precursor to the February 2020 event stated that Trondheim should not worry about flooding (\cite{baisotti_danger_2020}). However, the actual SLE was 215cm, which is exceptionally high, although this garnered little media attention. How information sources, particularly newspapers, influence stakeholders' awareness is discussed in the next section: Information Access about Climate and Place.
\paragraph{}
How memory is formed and how it impacts awareness and resilience is highly debated. There is evidence that institutions, particularly the media, have the potential to influence memory formation and recollection (\cite{de_guttry_expiry_2022}). There is also evidence that creating institutional memory of disasters due to natural hazards can help minimise the impact of the next one (\cite{de_guttry_expiry_2022}). However, none of the SLEs that have occurred in Trondheim since 1950 are considered here as a disaster, but simply an event. It is thus possible that the low severity of past SLEs has influenced memory of SLEs in Trondheim. 
\paragraph{}

The emotional relationship between people and place likely influences their memory of SLEs. It will also influence their views on risk to the place and the acceptable level of resilience of the place. A high emotional connection to a place may influence the fact that stakeholders often do not consider insurance as an acceptable form of resilience (Gonzalez-Riancho et al (2017) as cited in \cite{gerkensmeier_governing_2018}). Norwegian insurance companies do not seem to differentiate based upon the vulnerability of location to flooding (\cite{lujala_role_2020}). This may change in the future given the changing climate wth a corresponding change in the views of the population about risk and resilience.
\paragraph{}
Future research could consider the impact of place attachment, as discussed by \cite{ariccio_place_2021}, on resilience and acceptable levels of resilience. Also, the concept of memory anchoring could be used to increase social resilience to storm surges (\cite{de_guttry_expiry_2022}).  


\subsection{Information Access about Climate and Place}
Neither "awareness A" nor "awareness B" appeared dependent upon any of the information sources about place  (Table \ref{kw_test_info_place}). "awareness A" appeared dependent on family. The researcher is aware that they have discussed this thesis with many members of their family and that this could have influenced the results. For this reason, extra Kruskal Wallis tests, upon whether either forms of awareness were dependent upon survey access, were conducted. The results allowed for the acceptance of the null hypothesis that survey access (including due to connection to researcher) is independent of either determinations of awareness. Then, the independence of survey access and whether the information source of family was also checked. Again, these factors were deemed independent. These tests indicate that the fact that "awareness A" appears dependent on family can not easily be simplified into the narrative that this is only a factor for the researcher's family. For example, it is known that people accept information about the changing climate better from sources they trust (\cite{corner_a_principles_2018}). 
\paragraph{}
After the preliminary trials of the main survey, the first group who got access to the survey upon opening to the public were test engineers. This group was targeted to allow errors to be quickly noticed and fixed before the final results were impacted. The first 15 participants had access to 8 surveys, one of which included a typo and was missing education. The option of choosing "utddanning" (formal education) as where you learned about climate change was missing from two of the surveys. These errors were quickly remedied and a maximum of five \% of subjects could have been impacted. 
\paragraph{}
"awareness B" appears dependent on the information sources of "formal education" and "newspaper" for information about the changing climate. The small limitation on this factor is that for the very first few hours of data collection, one of the surveys was missing the option to select formal education. This was determined to impact a maximum of three responses and was quickly fixed. For this reason, this error in data collection method is deemed to have no major impact on the results. 
\paragraph{}
Of more significant impact is that the first day of data collection did not have peer-reviewed publications as an option for information source about changing climate. This was only added after it was written in by respondents as an answer they would like to have and was the subject of an email to the researcher.  The first 15 respondent did not have peer-reviewed publications as an option in the question about information sources for changing climate. However, two of these subjects had it added in later, because they wrote about this when given the opportunity for extra information to go to the researcher. To investigate the potential impacts of this, all the subjects who completed the survey before this was an option had this selection artificially added to see if it impacted the Kruskal Wallis Test results. The result from this attempt to minimise impact of this miss of information collection is that the null hypothesis was still accepted. Thus, that the information source of peer-reviewed publications about climate change is independent of awareness is still the conclusion made here, but with a greater degree of uncertainty.
\paragraph{}


\subsection{Community Membership}
An important limitation in the study is that no subjects responded that they were marine-workers as displayed in figure \ref{fig:community_membership}. Future research should target this group earlier in the data collection process. The lack of non-marine workers prevented analysis into whether their professional knowledge of the sea would translate into high determined awareness of SLEs. The use of emails was a major technique used to reach marine workers, the limitations of which are discussed in the next section on access to survey. 

\paragraph{}
The most popular community membership selected was "resident" with 70 out of 153 subjects choosing this option. Even with its high popularity, this result was lower than expected as it was predicted that mainly residents would be interested in taking part in this research. Perhaps a concern about a lack of non-residents influenced the participant split. The targeting of students may also have had an impact. By many definitions, the students who participated in this thesis could also be considered residents. However, the majority of subjects only selected one community membership. An important factor in interpreting figure \ref{fig:community_membership} is that there were only 221 responses for a subject group of 153. The mode number of responses to this question was one. That the majority of subjects only selected one response is also repeated in other questions where the subject could select multiple answers. Thus the lack of students who also selected resident may be due to the tendency to only select one answer. However, it may have been due to different conceptualisation of what it means to be a resident: were subjects considering themselves a resident only if they resided in the specific place they responded on, or did they consider themselves a resident if they lived in Trondheim. There is a potential that the conceptualisation of resident could have been linked to those who held Norwegian permanent residency. For future research, perhaps an alternative term could be used, or who is considered a resident defined within the survey. Whether the preamble to these types of questions needs to mention "please tick all that apply" is also something that should be considered if this technique is to be repeated. 
\paragraph{}

\subsection{Access to Survey}
While access to the survey does not appear to have influence on awareness of local knowledge, the results are important for the discussion on how to create a framework for determining resilience of a place. Almost all subjects only chose one access type. Just as with figure \ref{fig:community_membership} on community membership, figure \ref{fig:survey_access} on Access to Survey shows that the majority of respondents only chose a single answer. This is a limitation on the conclusions that can be drawn from this result. Whether every subject who had a personal connection to the researcher chose that access method can be debated. Again, terminology here is important: what does it mean to have a personal connection. Personal connection could be conceptualised as solely a family member or close friend, however it could include acquaintances, colleagues or even those with a shared group membership. For the determination of awareness which considered all five possible questions, "Family" is a significant factor. This means that this determination of awareness appears dependent on family as a sources of observation. This result is important in the context that 25 subjects have a personal connection to the researcher.
\paragraph{}
Beyond the researcher's personal connection to their family, another personal connection is to the members of Trondheim Kajakk Klubb. Members of Trondheim Kajakk Klubb were a targeted group due to their position as frequent water leisure users. The researcher is a member of this club which made it easier to access the members. However, it is believed that the members accessed the survey via social media, rather than access due to  membership of organisation or due to personal connection to researcher. This is a limitation on the interpretation of the results. Though there may have been some impact, the results indicates that if this thesiswas conducted in a location where the researcher does not have a connection would still yield results.
\paragraph{}
The least popular responses to the question on access to survey were "place of employment", "membership of organisation" and "email". Many firms have effective email filters in place which could have prevented the email from reaching stakeholders. Furthermore, the data was collected over the summer months, when people do not necessarily chose to access their email, thus limiting responses from these categories. On the other hand, there are more visitors and tourists available over the summer months. It is thought that if this survey was repeated in a different season, the split of respondents would vary. For this reason their is no conclusion given on the appropriateness of accessing stakeholders via "place of employment", "membership of organisation" or "email".
\paragraph{}
That 70 subjects accessed the survey via the poster, indicates that this method was successful in reaching stakeholders. The design of the poster may have influenced this. The poster is included in the appendix. To see the communication guidelines used, consult the methods section. It is likely that posters were more successful at attracting stakeholders when placed in certain locations. There was a consideration of making sure that each QR-code used to access the posters was linked to the precise location of the posters. This was decided against due to budgetary constraints but may be worth revisiting for future research.

\paragraph{}
The next most frequent choice on how subjects accessed the survey was via social media. When considering a framework to allow repeated measurements of resilience, posters and social media are recommended due to the number of responses. 
\paragraph{}

\section{Perceived Risk}
This survey will have influenced the subjects perception of risk: it is unlikely that 97 subjects would have selected storm surges as a risk to infrastructure before carrying out this survey. The impacts of citizen science and perception of risk are expanded upon in the discussion of framework, including the high levels of perception that shoreline instability is a risk to infrastructure in the research sites.

\subsection{Risk Perception - Personal Impacts of SLEs and Associated Flooding}
There were opportunities for subjects to give extra information on certain topics in the main survey, as can be seen in the appendix B. From these responses, narratives about perception of risk can be explored. For example, there was a very varied response to "How would flooding associated with SLEs in this area affect you? If you would like to give more details on this, please write below." These responses have been grouped into narratives depending on the impact level identified in an earlier question. 
\paragraph{}

Subjects who felt they would have "no impact" and then gave more details, either stated that they lived far from the sea, or that they would be moving away soon. This highlights an idea that if the SLE does not impact personal property or their daily activity then it will not affect them at all. Nuisance flooding was not considered an issue, nor how flooding could impact key infrastructure, for example transport connections. One subject gave an example of flooding of their normal car park, but did not identify this as an impact, because their car was not parked there during the event. However, it can be argued that their views upon this place have been altered. Memory of event was present and and quickly recalled enough for a short survey.  
\paragraph{}

Subjects who responded that flooding in this area would have "mild impact" and gave more information, highlighted a wider range of responses. The overall narrative was that flooding would not affect where they live, but may impact their family and their activities. One curious response was from an individual who said that because they lived on a higher floor, they would not be impacted. This perception gives a narrow view of impact and risk. An SLE may impact the buildings foundations and the subjects daily activities, even though they are on a higher floor. This continues the narrative that SLEs may impact activities, but as long as it does not obviously damage my private property, the impact can be considered mild. This raises questions about what it means to experience a flooding hazard. 

What it means to have direct experience of flooding can have varied definitions. \cite{whitmarsh_are_2008} defines direct flooding experience as those who have experienced forms of flood damage to their home, garden or vehicle within the last half decade. This is a tight definition of direct experience and focuses on ownership. People who do not own a home, garden or vehicle can still be impacted by flooding and their experience may be very different. Here, flooding impact is conceptualised in a broader sense and includes impact to activities, sense of place and communities.   
\paragraph{}
It is commonly assumed that direct experience makes people more concerned about climate change and its potential personal impacts, but this does not always appear to be the case (\cite{lujala_role_2020}). Perhaps some of this variation is due to variation in what is defined as direct experience. Further research could look into differing types of direct experience. 
\paragraph{}
Only when subjects responded with medium impact or greater, was a negative emotional response expressed. Subjects discussed how flooding will affect places where they stored things, particularly at the harbour. Subjects who ranked flooding in this area as significant, gave similar response. The overall narrative expressed from the extra information given here is that subjects appear to believe that if SLEs do not affect their private property, it will not affect them. 
\paragraph{}
None of the subjects who gave extra information mentioned insurance as a factor in their perception of personal impact.  Gonzalez-Riancho et al (2017 as cited in \cite{gerkensmeier_governing_2018}) state that stakeholders for the Wadden Sea Region, a similarly developed area in Northern Europe with a longer history of negative flooding impacts from SLEs, reject that insurance can be considered as a form of disaster risk management. Perhaps the self-selecting stakeholders for Trondheim have similar views. How economists view the role of insurance in the case of resilience to flooding often does not correspond with other stakeholders, particularly residents (\cite{gerkensmeier_governing_2018}).
\paragraph{}

\cite{lujala_climate_2015} discusses the unexpected results that living in area with high physical vulnerability worsened by the changing climate did not impact peoples' concern level about climate change or natural hazards in that area. However, they also highlight that participants who experience direct negative experience (damage to private property) were more concerned about the changing climate. This finding is in line with \cite{whitmarsh_are_2008}. These findings imply that only when a place has a disaster which causes damage to private property do these individuals become more concerned about the changing climate or prepare themselves for increasing risk of flood damage. An alternative interpretation of \cite{lujala_climate_2015} is that participants may have been accurately ranking their personal risk, due to Norway's privileged global position. The results from this paper indicate that while there may be a decrease in social, technological and natural systems resilience, none of the systems are projected to fail completely. In general, it is projected that few people will die directly due to SLEs in Trondheim over the coming decades.
\paragraph{}


There is widespread insurance cover of buildings in Norway against damage from natural hazards (\cite{lujala_role_2020}). Households with fire insurance are required to also have insurance for the damage caused by landslides, storms and flooding. The cost of this is connected to the insurance rate for fire damage and is not dependent on the location of the house (\cite{lujala_role_2020}). The role of insurance and the welfare state in the perception of risk in Trondheim is likely significant (\cite{lujala_role_2020}), but this was not mentioned by any of the subjects regardless of their perception of the potential personal impact of flooding in these places. Future research could include questions to gain a greater understanding of how stakeholders view the role of the welfare state or insurance policies in their perception of this risk. 

\subsection{Risk Perception - Beyond SLEs}
Subjects were given the opportunity to highlight other risks to people and infrastructure which were not mentioned in the main survey. They highlighted risks associated with the natural systems and the interplay between the technological and social system. The concern about the risk of land erosion is displayed in the quote below from a subject who ranked the impact of flooding their place as significant.

\begin{itemize}
    \item "A few days ago, I saw a video of some part of Norway where an entire chunk of land just went under the water, I guess due to erosion, and I would assume that storm surges have an effect on erosion. That video was horrifying and I keep thinking, if something like that to happen here in Trondheim, which part of coast would it take away. I would love to buy/rent a house near the coast, but after watching that, I am not so sure."
\end{itemize}
\paragraph{}

At a surface level, this quote highlights a fear of flooding, and how this impacts spending decisions and sense of place. Taking it further, however, the fear displayed is of "land falling away". They are concerned about the land erosion caused perhaps by SLEs.
\paragraph{}
This fear of "land falling away" was brought up by several subjects at multiple points when given the opportunity to broaden the survey. For example, when asked "If you would like to give other examples of risk in this area, please write here." by far the most common response was risk of landslide - often specifically due to quick clay. This response was mainly given by those taking the survey in Norwegian. 
\paragraph{}
Stakeholders awareness of SLEs appears dependant on information sources about the changing climate. Awareness appears dependant on the stakeholders use of newspapers, family and formal education. Quick clay and landslides get media attention in Norway and this may impact awareness and resilience for these natural hazards. This could be explored in a conduction of multi-risk resilience study for Trondheim.  
\paragraph{}
The other responses were about risks associated with the interplay between technological and social systems. The most common of these responses focused around lack of housing. This was sometimes discussed in the context of people having to move due to changing climate, including rising SLEs, but not always.  As earlier mentioned, many respondents stated that they will minimise personal impact of flooding of SLEs by not residing along the coast.  As stated in the background, Trondheim has 34.6 \% of the land use in its coastal area influenced by buildings (\cite{engebakken_construction_2022}), a significant \%age which are residences. The impact to housing in Trondheim is likely not that significant within the next 70 years, but the impact in the desire to live there may be greater. This push inland could have significant economic, social and ecological impacts in Norway. The impact to the cultural systems aspect of social systems is particularly likely to change, but if this is viewed as creating a new normality, rather than as an inability to return to the previous normal, this can still be considered as having high levels of resilience (\cite{cutter_place-based_2008} and \cite{cutter_community_2020}).
\paragraph{}




\section{Awareness} \label{discuss-aware}

\subsection{Determination of Subjects' Awareness}
The results from the survey for both the variables and factors were skewed (figures \ref{fig:interest_level_SLE} to \ref{fig:aware_all_edited_photo}). The results from the histogram analysis is that no variable of awareness appeared normally distributed. Upon basic transformation, including logarithmic and cubing, it was highly skewed. This impacted the data analysis which could be used in the determination of awareness and in turn which factors impacted awareness. 

\paragraph{}
The key result from the pilot survey and focus group is that determining awareness from surveys was an appropriate method for use in a determination of social resilience to SLEs in a place. Also, that utilisation of maps was not appropriate for the research sites chosen due to the low level of projected changes and that simulated images of visual extremes were a better choice.
\paragraph{}
Information access and community membership appear to be factors in the determined awareness of subjects. Over half of subjects select storm surges as a perceived risk to people and infrastructure. Subjects appear to be somewhat aware of the changing SLEs. The format in which questions of SLEs are asked appears to have an impact. 

\paragraph{}
The determination of awareness displayed in figure \ref{fig:aware_all_edited_photo} was chosen as the most appropriate variable due to the fact that only awareness questions with a simulated water level picture were included, as can be seen in the appendix. The other determination of awareness included the results from two questions which were based purely off numeric answers which gave very skewed results. When given just numeric responses, the accurate rate was non significant. For example, only 3 subjects out of 153 gave the correct answer for projections of sea level change in the last 30 years as shown in figure \ref{fig:slr_past}.  This is unlikely given that it was multiple choice with only 7 potential answers (if split evenly then would have 21 subjects choosing this response). Figure \ref{fig:slr_future} of subjects predictions for how sea level will change in the next 30 years has more correct responses. This was only mildly significant, which combined with the lack of correct answers as shown in figure \ref{fig:slr_past} was considered a valid reason to exclude the answers to both of these questions for creating the variable of awareness. Awareness is considered here as a continuous variable which is not independent. Rather than simply asking subjects about their awareness, there was an attempt to determine the awareness. The questions used were not easy. For this reason, having managed to answer some of the questions correctly was enough for the subject to be classed as aware.

\paragraph{}
Another determination of awareness was considered which only includes results from questions on current storm surge and current high tide, excluding results of the question of the 20 year storm surge in 2090. This was considered as a potential baseline variable for current awareness to SLEs as different from awareness to future SLEs. However, the variable summary statistics are similar. For this reason, Kruskal Wallis Tests were run with all factors against these two variables. Not all of these were presented as they were not all relevant for answering the hypotheses or research questions. 

\paragraph{}
The Natural Hazard to Risk to Behaviour Pipeline (figure \ref{fig:hazard_to_behaviour}) shows how awareness influences behaviour. This view of awareness of natural hazard risk as being filtered through perceived distance, personal experience, values and resources dictated the search for factors which influenced awareness. 



\subsection{Factors Affecting Subjects' Awareness}
As can be seen in the hypothesis testing summary, very few factors dependent upon awareness were found. Those factors discovered may have a positive or negative relationship with awareness. Awareness was found to be dependent upon subjects using newspapers to learn about the changing climate. From these results, we cannot state that increasing newspaper readership increases awareness of SLEs. In fact, it could be the opposite. Future research can explore the relationship between these factors and awareness, perhaps by utilising linear mixed effect models.
\paragraph{}
Awareness is considered a key facet of local knowledge within the framework outlined in figure \ref{fig:social_resilience}. Other facets impacting community resilience and therefore social system resilience are institutions, social networks and resources. The institutions of national government, local municipalities, organisations, schools and media publications were considered within this analysis of Trondheim's resilience to SLEs. Future research could focus on better including social networks outwith institutions and the spread of financial resources. For disaster risk management in Norway, responsibility lies with the municipalities and the Ministry of Justice and Security; these institutions have high levels of resources. Trondheim, by global standards, has high levels of resources. This will have affected the subjects' views of risk and resilience as discussed in the sections about risk perception. 
\paragraph{}
The determination of awareness and the search for factors of awareness was done in such a way to attempt to support the development of a framework for quickly and cheaply determining resilience. Edited photographs, posters, social media posts and online surveys are less costly than conducting interviews, stakeholder workshops or utilising virtual reality technology. \cite{gerkensmeier_governing_2018} had the same results for qualitative survey as with stakeholder workshops in their investigation upon storm surges. Future research may wish to trial both methods in Trondheim to see if this is also the case here. Another reason to include them digital surveys in the understanding of resilience is that they can be used to access parts of the population who would not take part in stakeholder workshops. However, if attempting to analyse Trondheim's overall resilience rather than single-risk resilience, stakeholder workshops may be more useful.


\section{Research Question 1 - How resilient will Trondheim be to SLEs during 2022 to 2050 and 2050 to 2100? } \label{RQ1-findings}
From the literature review, Trondheim's natural systems of resilience and technological systems of resilience can be considered very high for the period of 2022 to 2050. From the results of the survey,  we can say that social systems are somewhat resilient for this same period. Therefore, total projected resilience to SLEs for this period can be described as high. However, this does not mean that there is no potential economic, cultural or social impacts of SLEs. Only that it is projected that the daily activities occurring in Trondheim will quickly return to normality after the likely SLE events during this period.
\paragraph{}
The awareness determined for SLEs in 2090 is higher than the awareness for 2022. The determination of projected social system resilience for the period of 2050 to 2100 is high. The technological and natural system resilience is also determined as high. Therefore, the overall projected resilience for Trondheim to SLEs during 2050 to 2100 is high. However, projected resilience is highly dynamic, meaning that the resilience of a place will change over time. There are requirements from the Sendai framework for and the United Nations sustainable developments goals (UN SDGs) 11 and 13 that cities resilience to natural hazards should increase (\cite{gonzalez-riancho_storm_2017}). To see if this is occurring, repeated measurements of resilience are required. 

\paragraph{}
This requirement of increasing resilience brings up the question of what level of resilience should a place have. Determining what level is acceptable to stakeholders is suggested as a method of answering this question by \cite{gerkensmeier_governing_2018}. The four research sites of Skansen, Grillstad, Brattora and Nidelva represent the most vulnerable places in Trondheim to SLEs. This focus on vulnerable sites helps minimise the chance of overestimating resilience. The results from the survey indicate that the factor of Place had no impact on how aware subjects were to SLEs, even though each of these places had different levels of natural and technological system resilience. 
\paragraph{}
This determination of Trondheim's resilience to SLEs is backed up by \cite{opach_seeking_2020}. They conducted a cluster analysis of community resilience to natural hazards for Norway and ranked Trondheim as having high economic, housing and infrastructure resilience. They were designated with lower institutional and community capital, decreasing social system resilience as described here. However, it can be argued that the way these valuations for resilience were determined were biased against cities and favoured less populous areas.



\section{Research Question 2 - Are Stakeholders aware about changes to SLEs?} \label{RQ2 - findings}

The findings here are that stakeholders are somewhat aware about changes to SLEs. Whether this is an acceptable level of awareness is dependent on how resilient Trondheim should be. If Trondheim is to improve its resilience, improving social system resilience should be considered as a method, as it has been found that technological and natural systems already display high resilience. Figure \ref{fig:aware_all_edited_photo}5.24 shows that when shown the information about SLEs using edited photographs and numeric valuations, 13 \% of subjects answered all questions correctly; 40 \% answered one correctly; almost 40 \% answered two out of the three correctly; and only 7 \% of subject answered none of the questions correctly. The questions to determine awareness were deliberately difficult as a spread of the results of awareness was desired. If all subjects were classed as aware, it would be impossible to determine which factors awareness is dependent upon. 
\paragraph{}
Stakeholders in Trondheim do not appear aware of the changing SLE over the past 30 years. Very few responses highlighted that sea levels in Trondheim have in fact decreased over the last decades. This rises questions about memory. The level of difference that is required for individuals to notice and remember is particularly important here. There has been under 30 cm change in sea levels in Trondheim in the last 30 years. It is proposed that this variation was simply too small for people to notice, particularly with the continuous impacts of weather and tide. 
\paragraph{}

\section{Research Question 3 - What factors impact stakeholders' awareness of SLEs? } \label{RQ3 - finding}
 As outlined in the results chapter, there were differences discovered from the awareness utilising all questions and the awareness determined from questions using edited photographs. 
\paragraph{}

In investigating what factors impact stakeholders awareness, two determinations of awareness were used. "awareness A", which included all five questions from the survey upon awareness and "awareness B", which only included the three questions which utilised the edited photographs showing simulated SLEs. The impact of numeric versus visual information upon SLEs is discussed in Chapter 6.
\begin{enumerate}
    \item Community membership does appear to impact stakeholders awareness of SLEs. Awareness appears dependent on the factor of residency, meaning we accept the hypothesis that residents are aware. The rest of the hypotheses dependent on community membership are rejected, apart from marine workers, which has no result as there were no responses to analyse.
  
    \item Stakeholders awareness of SLEs did not appear dependent on the factors of local knowledge considered. Awareness does not appear dependent on Interest Level, primary knowledge of places on reclaimed land, length of knowledge, the number of information sources about place. Whether subjects responded in Norwegian or English appears to have no impact on awareness. Stakeholders awareness does not appear dependent on their information sources about place. 

    \item  Stakeholders information sources about the changing climate does appear to impact stakeholders awareness of SLEs. Stakeholders awareness appears dependent upon their use of the information sources of family, peer reviewed publications and formal education. The awareness does not seem to be dependant on the other sources of information about the changing climate or the number of sources used. Stakeholders awareness does appear dependent on their concern about climate change. Stakeholders awareness is not dependent upon their prediction that they will be impacted by flooding from SLEs.
\end{enumerate}
\paragraph{}
How few factors were deemed to have impact on awareness was surprising. This may be due to the lack of awareness found.  Future research could flip this research question and hypotheses round and attempt to discover which factors make stakeholder's unaware of SLEs. 








