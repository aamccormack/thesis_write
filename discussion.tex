% LIMITS OF SURVEYING AS TECHNIQUE
% COMMUNICATION RISK - REACHING OUT TO SUBJECTS
%LIMITS OF EDITED PHOTOS
% HOW TO CREATE A FRAMEWORK
%LIMITS OF RESULTS - E.G. CODING /STATS LIMITS
%LIMITS OF SURVEYING AS TECHNIUE 
%LACK OF INTERVIEWS
%PEOPLE ONLY TICK ONE BOX WHEN ASKED CERTAINS QS E.G. COMMUNITY MEMEBERSHIP
%LESSONS LEARNED FOR REPEATING

\chapter{Discussion of Results}

\section{Overall Results}

\section{Research Question 1}
%i.e. lit review says trondheim is pretty resileint
%but less socially resilient or economically resilient

\section{Research Question 2}
%people are somewhat aware


\section{Research Question 3}
%more difficult as not so many aware
%could try and reverse it - what makes people unaware 
%interest level doesn't appear to be a factor in awareness


The choice which format method to use while asking about sea level changes is likely to have an influence on how participants perceive this change. For example considering change as metres in height vs area of land influenced may results in different perceptions of the changing risks and potential impacts.  By sticking firmly in the realm of historical fact and scientific models with these questions there has been an attempt to investigate awareness of SLEs rather than perception. 

\section{Visualisation of sea level change}
Discuss exclusion of questions "How much do you think the sea level will change in the next 30 years?" and "How much do you think the sea level has changed here in the past 30 years?". 

Especially why focus on answers which were picture based versus answers that were number based


%EDUCATION AS AN OPTION WAS ONLY ADDED AFTER day 1
%as was peer reviewed articles for the norwegian options
%so first 20 respondents did not have that option

\section{Awareness vs Assumed Awareness}
Discuss lack of awarenes, particularly in groups who could be assumed to be aware of risk

e.g. graphs - interest level vs determined awareness (ss-aware-all or ss-aware)

do highlight lack of marine workers who responded while disucssing this. 

\section{Personal Impacts of Sea Level Extremes and associated Flooding}
There was opportunity for subjects to give additionary information on certain topics as can be seen in the appendix. From this certain narratives about perception of risk can be explored. For example their was a very varied response to "How would flooding associated with sea level extremes in this area affect you? If you would like to give more details on this, please write below." This can be grouped into narratives depending on the impact level the subjects gave to the previous question.

Subjects who felt they would have no impact who gave more details either stated that they lived far from the sea, or that they would be moving away soon. This highlights an idea that if the sea level extreme does not impact items they belong or their daily activity that it will not affect them at all. Nuisance flooding was not considered an issue, nor how flooding could impact key infrastructure for example transport connections. One subject gave details of how flooding did impact where they normally park their car, but as it wasn't parked during this event then they weren't impacted. 

Subjects who responded that flooding in this area would have mild impact and gave more information gave a wider range of responses than those who felt it would not impact them. The overall narrative was that flooding won’t affect where they live, but may impact their family and their activities. One curious response was from an individual who said that because they lived on a higher floor they would not be impacted. Living on a higher floor may provide some protection but in a extremely high sea level could still have impacts on daily activities. This continues the narrative that sea level extremes may impact my activities but as long as it doesn't negatively impact my possessions the impact can be considered mild. This raises questions about what it means to experience a flooding hazard. What it means to have direct experience can have varied defintions. 
\cite{whitmarsh_are_2008} defines direct flooding experience as those who have experienced forms of flood damage to their home, garden or vehicle within the last half decade. This is a particularly tight definition of direct experience and focuses on ownership. People who do not own a home, garden or vehicle can still be impacted by flooding and their experience may be very different.  

It is commonly assumed that direct experience makes people more concerned about climate change and its potential personal impacts\cite{lujala_role_2020}. However this does not always appear to be the case \cite{lujala_role_2020}. Perhaps some of this variation is due to variation in what is defined as direct experience. Further research could look into differing types of direct experience. 

Only once subjects responded with medium impact did a negative emotional response get expressed. Subjects discussed how flooding will affect places where they stored things, particularly at the harbour. Subjects who ranked flooding in this area as Significant impact gave similar response. The overall narrative expressed from the additonal information given here is that subject seem to think if it doesn’t affect the metres of where they live it wont affect them. 

\section{Information sources and fear of natural hazard }

One subject who ranked the impact of flooding as the area as significant and gave details why answer is worth analysing further as it highlights another view brought up. 
Quote from one subject: 
"A few days ago, I saw a video of some part if Norway where an entire chunk of land just went under the water, I guess due to erosion, and I would assume that storm surges have an effect on erosion. That video was horrifying and I keep thinking, if something like that to happen here in Trondheim, which part of coast would it take away. I would love to buy/rent a house near the coast, but after watching that, I am not so sure."

At a surface level this quote highlights a fear of flooding and that they like other subjects who labelled their impact as no impact will avoid living near the coast in part due to fear of flooding. But taking it further the fear here isn't simply of sea level extremes but of land falling away. 

This fear of land falling away is brought up by several subjects at multiple points when given the opportunity to broaden the survey. For example when asked "If you would like to give other examples of risk in this area, please write here." by far the most common response was risk of landslide - often specifically due to quick clay. This response was mainly given by those taking the survey in Norwegian. 

*EXPAND**quick clay & landslides get media attention - how does this impact awareness and resilience...

Other responses focused around lack of housing, because people will have to move. This view is the flip side of the many respondents who stated that they will minimise personal impact of flooding due to sea level extremes by not residing along the coast. 
Trondheim has 34.6 percent of the land use in its coastal area influenced by buildings \cite{engebakken_construction_2022}, a significant percentage which are residences. The impact to housing in Trondheim is likely not that significant within the next 70 years, but the impact in the desire to live their may be significant. This push inland could have significant economic, social and ecological impacts in Norway. 