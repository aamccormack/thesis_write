% LIMITS OF SURVEYING AS TECHNIQUE
% COMMUNICATION RISK - REACHING OUT TO SUBJECTS
%LIMITS OF EDITED PHOTOS
% HOW TO CREATE A FRAMEWORK
%LIMITS OF RESULTS - E.G. CODING /STATS LIMITS
%LIMITS OF SURVEYING AS TECHNIUE 
%LACK OF INTERVIEWS
%PEOPLE ONLY TICK ONE BOX WHEN ASKED CERTAINS QS E.G. COMMUNITY MEMEBERSHIP
%LESSONS LEARNED FOR REPEATING

\chapter{Discussion of Results}

\section{Overall Results}

\section{Research Question 1}
%i.e. lit review says trondheim is pretty resileint
%but less socially resilient or economically resilient

\section{Research Question 2}
%people are somewhat aware


\section{Research Question 3}
%more difficult as not so many aware
%could try and reverse it - what makes people unaware 
%interest level doesn't appear to be a factor in awareness


The choice which format method to use while asking about sea level changes is likely to have an influence on how participants perceive this change. For example considering change as metres in height vs area of land influenced may results in different perceptions of the changing risks and potential impacts.  By sticking firmly in the realm of historical fact and scientific models with these questions there has been an attempt to investigate awareness of SLEs rather than perception. 

\section{Visualisation of sea level change}
Discuss exclusion of questions "How much do you think the sea level will change in the next 30 years?" and "How much do you think the sea level has changed here in the past 30 years?". 

Especially why focus on answers which were picture based versus answers that were number based


%EDUCATION AS AN OPTION WAS ONLY ADDED AFTER day 1
%as was peer reviewed articles for the norwegian options
%so first 20 respondents did not have that option

\section{Awareness vs Assumed Awareness}
Discuss lack of awarenes, particularly in groups who could be assumed to be aware of risk

e.g. graphs - interest level vs determined awareness (ss-aware-all or ss-aware)

do highlight lack of marine workers who responded while disucssing this. 