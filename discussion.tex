% LIMITS OF SURVEYING AS TECHNIQUE
% COMMUNICATION RISK - REACHING OUT TO SUBJECTS
%LIMITS OF EDITED PHOTOS
% HOW TO CREATE A FRAMEWORK
%LIMITS OF RESULTS - E.G. CODING /STATS LIMITS
%LIMITS OF SURVEYING AS TECHNIUE 
%LACK OF INTERVIEWS
%PEOPLE ONLY TICK ONE BOX WHEN ASKED CERTAINS QS E.G. COMMUNITY MEMEBERSHIP
%LESSONS LEARNED FOR REPEATING

%size of city plus no major disaster impacts results and discussion
%seasonal aspect

%remember survey was difficult - can say quite aware even if only got 2 correct

%issue comparing norwegian and english surveys - word just dont mean the same - direct translation is impossible -cultural differences when viewing risk

\cite{cutter_community_2020} discusses difference between insurance and resilience - how economic resilience can be impacted at lower levels of flooding impact

\chapter{Discussion of Results}


\section{Research Question 1 - How resilient will Trondheim to be to sea level extremes during 2022 to 2050 and 2050 to 2100? }
%i.e. lit review says trondheim is pretty resileint
%but less socially resilient or economically resilient
From the literature review Trondheim's natural systems of resilience and technological systems of resilience can be considered very high for the period of 2022 to 2050. The social systems of resilience for this period are only considered as medium as the awareness determined for the current sea level extremes are only somewhat. Overall Trondheim's resilience to sea level extremes can be considered high, due to the limited damage potential and the high level of resources available to deal with it.
\paragraph{}
The awareness determined for sea level extremes in 2090 is higher than the awareness for 2022. This allows for the projected social system resilience for the period of 2050 to 2100 to be high. The technological system resilience is also considered as high for this period and the natural system resilience is considered medium. However projected resilience is an highly dynamic meaning that the resilience of a place will change over time. There is requirements from goal 11 and 13 that cities resilience to natural hazards should increase. To see if this is occurring repeated measurements of resilience is required. 

\paragraph{}
This requirement of increasing resilience brings up the question of what level of resilience should a place have? What level is acceptable to stakeholders is suggested as a method of answering this question by \cite{gerkensmeier_governing_2018}. The four research sites of Skansen, Grillstad, Brattora and Nidelva represent the most vulnerable places in Trondheim to sea level extremes. Taking the extreme potentials of sea level extremes, helps minimise the chance of overestimating the resilience Trondheim has to sea level extremes. The results from the survey indicate that the place had no impact on how aware subjects were to sea level extremes, even though each of these places had different level of natural and technological system resilience. 




However how high should the resilience to sea level extremes of Trondheim be? The UN SDGs goal 11 and 13 demand an increase in cities resilience to natural hazards, as does the Sendai framework *add citation*.  Perfect resilience is impossible, one way of determining an acceptable level of resilience is by asking stakeholders what is an acceptable level \cite{gerkensmeier_governing_2018}. 

\section{Research Question 2}
%people are somewhat aware


\section{Research Question 3}
%more difficult as not so many aware
%could try and reverse it - what makes people unaware 
%interest level doesn't appear to be a factor in awareness which contrast with 


The choice which format method to use while asking about sea level changes is likely to have an influence on how participants perceive this change. For example considering change as metres in height vs area of land influenced may results in different perceptions of the changing risks and potential impacts.  By sticking firmly in the realm of historical fact and scientific models with these questions there has been an attempt to investigate awareness of SLEs rather than perception. 

\section{Visualisation of sea level change}
Discuss exclusion of questions "How much do you think the sea level will change in the next 30 years?" and "How much do you think the sea level has changed here in the past 30 years?". 

Especially why focus on answers which were picture based versus answers that were number based


%EDUCATION AS AN OPTION WAS ONLY ADDED AFTER day 1
%as was peer reviewed articles for the norwegian options
%so first 20 respondents did not have that option

\section{Awareness vs Assumed Awareness}
Discuss lack of awarenes, particularly in groups who could be assumed to be aware of risk

e.g. graphs - interest level vs determined awareness (ss-aware-all or ss-aware)

do highlight lack of marine workers who responded while disucssing this. 

\section{Risk Perception - Personal Impacts of Sea Level Extremes and associated Flooding}
There was opportunity for subjects to give extra information on certain topics as can be seen in the appendix. From these responses narratives about perception of risk can be explored. For example their was a very varied response to "How would flooding associated with sea level extremes in this area affect you? If you would like to give more details on this, please write below." This has been grouped into narratives depending on the impact level the subjects gave to the previous question.
\paragraph{}

Subjects who felt they would have no impact who gave more details either stated that they lived far from the sea, or that they would be moving away soon. This highlights an idea that if the sea level extreme does not impact items they own or their daily activity that it will not affect them at all. Nuisance flooding was not considered an issue, nor how flooding could impact key infrastructure for example transport connections. One subject gave details of how flooding did impact where they normally park their car, but as it wasn't parked during this event then they weren't impacted. 
\paragraph{}

Subjects who responded that flooding in this area would have mild impact and gave more information gave a wider range of responses than those who felt it would not impact them. The overall narrative was that flooding won’t affect where they live, but may impact their family and their activities. One curious response was from an individual who said that because they lived on a higher floor they would not be impacted. Living on a higher floor may provide some protection but in a extremely high sea level could still have impacts on daily activities. This continues the narrative that sea level extremes may impact my activities but as long as it doesn't negatively impact my possessions the impact can be considered mild. This raises questions about what it means to experience a flooding hazard. What it means to have direct experience can have varied definitions. 
\cite{whitmarsh_are_2008} defines direct flooding experience as those who have experienced forms of flood damage to their home, garden or vehicle within the last half decade. This is a particularly tight definition of direct experience and focuses on ownership. People who do not own a home, garden or vehicle can still be impacted by flooding and their experience may be very different.  
\paragraph{}
It is commonly assumed that direct experience makes people more concerned about climate change and its potential personal impacts\cite{lujala_role_2020}. However this does not always appear to be the case \cite{lujala_role_2020}. Perhaps some of this variation is due to variation in what is defined as direct experience. Further research could look into differing types of direct experience. 
\paragraph{}
Only once subjects responded with medium impact did a negative emotional response get expressed. Subjects discussed how flooding will affect places where they stored things, particularly at the harbour. Subjects who ranked flooding in this area as Significant impact gave similar response. The overall narrative expressed from the extra information given here is that subjects appear to believe that if sea level extremes do not affect the metres of where they live it wont affect them. 
\paragraph{}
None of the subjects who gave extra information mentioned insurance as a factor in their perception of personal impact level of flooding due to sea level extremes. GonZalex-Riancho et al, 2017 in \cite{gerkensmeier_governing_2018} says that stakeholders for the Wadden Sea Region a similarly developed area in Northern Europe, who has a longer history of negative flooding impacts from sea level extremes, reject that insurance can be considered as a form of disaster risk management. Perhaps the self-selecting stakeholders for Trondheim have similar views. How economists view the role of insurance in the case of resilience to flooding often does not correspond with other stakeholders, particularly residents \cite{gerkensmeier_governing_2018}.

There is widespread insurance of buildings in Norway against damage from natural hazards \cite{lujala_role_2020}. Households with fire insurance are required to also have insurance for the damage caused by landslides, storms and flooding. The cost of this is connected to the insurance rate for fire damage and is not dependent on the location of the house \cite{lujala_role_2020}. The role of insurance and the welfare state in the perception of risk in Trondheim is likely significant \cite{lujala_role_2020}, but this was not mentioned by any of the subjects regardless of their perception of the potenital personal impact of flooding in these places. Future research could include questions to gain a greater understanding of how stakeholders view the role of the welfare state or insurance policies in their perception of this risk. 






\section{Information sources and fear of natural hazard }
Mentioned several times by subjects in the extra information responses is the risk of the from land falling away. This concern about this risk is highlighted in the quote below from a subject who ranked the impact of flooding the area as significant.

\begin{itemize}
    \item "A few days ago, I saw a video of some part if Norway where an entire chunk of land just went under the water, I guess due to erosion, and I would assume that storm surges have an effect on erosion. That video was horrifying and I keep thinking, if something like that to happen here in Trondheim, which part of coast would it take away. I would love to buy/rent a house near the coast, but after watching that, I am not so sure."
\end{itemize}
\paragraph{}

At a surface level this quote highlights a fear of flooding and that they like other subjects who labelled their impact as no impact will avoid living near the coast in part due to fear of flooding. But taking it further the fear here isn't simply of sea level extremes but of land falling away. 
\paragraph{}

This fear of land falling away is brought up by several subjects at multiple points when given the opportunity to broaden the survey. For example when asked "If you would like to give other examples of risk in this area, please write here." by far the most common response was risk of landslide - often specifically due to quick clay. This response was mainly given by those taking the survey in Norwegian. 
\paragraph{}
*EXPAND**quick clay and landslides get media attention - how does this impact awareness and resilience...
\paragraph{}

Other responses focused around lack of housing, because people will have to move. This view is the flip side of the many respondents who stated that they will minimise personal impact of flooding due to sea level extremes by not residing along the coast. 
Trondheim has 34.6 percent of the land use in its coastal area influenced by buildings \cite{engebakken_construction_2022}, a significant percentage which are residences. The impact to housing in Trondheim is likely not that significant within the next 70 years, but the impact in the desire to live their may be significant. This push inland could have significant economic, social and ecological impacts in Norway. 
\paragraph{}




\section{notes from reading may want to expand upon - to delete}

Experience types (of flooding) - damaged belongings, survived, nuisance, direct observation, family/friend observation, heard about e.g. media
