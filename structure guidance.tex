Thesis structure idea 
Note: 35,000 –44,000 words in length (c.100-125 pages) 
The number of words is excluding the summary, preface, the contents, figures, and tables lists, references/literature lists, and appendixes. The title page should be followed by a summary in English, maximum 1 page/approx.350 words. 
The main text (body text) should be typed in 12 ptTimes New Roman font with 1.5 line 
spacing

Abstract
500 WORDS  - not included in word count
Abstract in Norwegian - would be nice if I have time
Preface
300 words - not included in word count
Intro / foreword 
5 PAGES
Background of locations
3 PAGES
Theory
10 PAGES
Method
2 PAGES
Results
10 PAGES 
Discussion of results
Results related to research qs
30 PAGES
Discussion of framework 
Results related to the reproducibility of this technique
40 PAGES 
Conclusion 
1000 WORDS 2 PAGES
Bibliography

Abstract
Will write last
Intro / foreword

Norway’s resilience to sea level extremes is and has been a dynamic process. To better protect coastal infrastructure and populations an understanding of how this resilience is changing is required. Resilience is impacted by many factors which can be considered as falling within three key systems. The natural, technological and social systems. These interplay to alter the level of resilience and risk within places. Sea level extremes are and extremely localised hazard in Norway and the understanding of this risk needs to be at a regional rather than national or global level. 
Changes to Norways Reslience to SLEs
•	Climate
•	Social 
•	Geographic/ geological
•	Link to examples of change of risk
•	Link to global changes in Sles
Project purpose
•	Determine resilience to sea level extremes in 4 key places in Trondheim
•	Assist creation of framework for quickly and cheaply determining a places resilience to changing sea level extremes
•	Connect local knowledge to academic knowledge
Link to un sdgs – 9, 11, 13

Theory
What is a SLE?
What is Resilience?
•	What is resilience
•	Projecting resilience
•	Natural + social + technical systems which impact resilience
Why Resilience of Place (rather than community)
•	DROP model Cutter
Make sure key terms are defined and explained
•	Resilience
•	Storm surge
•	Uplift
•	High water
•	Tide
•	Place
•	Disaster / event
•	20-year storm surge
•	Risk
•	vulnerability

Locations of Project Background

Trondheim and SlEs
•	2.06 m high water (NN2000)
•	Maximum observed in Trondheim is 2.60m in 1971
•	City plans for 4.87m by 2100
•	More details on chosen research areas
•	Include maps here 
•	Social demographic stats
•	Risk level stats
•	
location	Brattøra	Grillstad	Skansen 	Nidelva
PV 70 years ago	high	high	medium	Low
PV now	 medium	 medium	 low	 low
PV in 70 years	 high	 high	 medium	 medium
Dominant use	Office space and harbour	Residential	Recreational and industry	Residential and commerical
Land type	Reclaimed land	Reclaimed land	Managed coastline 	Altered tidal river
protection	Harbour wall	Harbour wall	Harbour wall, placed rocks	inland
 	 	 	 	 

o	Stakeholders
o	I.e. list all community membership options I asked
o	Plus the others people wrote in

Citizen Science
•	Pros and cons
•	Why important for this project
•	How I did it with respect to communities
Method Considerations
•	Pilot survey
•	Focus group
•	Results
Methodology
Method of gathering data
o	Website details
o	Poster details
o	Email details
o	Social media details
o	Importance of doing these to a high standard to maximise subject numbers and wide range of subject
o	And to inspire trust especially when this is a emotionally tricky subject
•	Method of analysing data
o	Excel
o	R
o	Maps?
Results
Discussion of results
Results related to research qs
•	Descriptive statistics
•	Results R1, R2, R3 
•	Method consideration
o	Survey limitations
•	
Discussion of framework 
Results related to the reproducibility of this technique
Conclusion 
bibliography

