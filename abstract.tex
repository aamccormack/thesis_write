
%Abstract (one page, c.350 words)
%Short summary of the thesis
%• what you investigated
% how you did it
%what you found out


\section{Abstract}

Improving the resilience of human settlements is required by United Nations Sustainable Development Goals (UN SDG's) 11 and 13. As a coastal country, the resilience to sea level extremes is of great importance to Norway. Previous studies have focused on national or global level risk, however sea level extremes are highly localized  and the understanding of risk needs to be at a regional and local levels. The purpose of this thesis is to determine resilience to sea level extremes in 4 key sites in Trondheim and contribute to the creation of a framework for quickly and inexpensively determining a place's resilience to changing sea level extremes. Projected resilience is determined from the interplay of natural, technological and social systems, which can all vary in space and time. The most dynamic factors are those within the social systems of resilience and as such are the focus in this thesis. Stakeholder surveys conducted in Trondheim coastal neighbourhoods indicate better awareness of the risk of sea level extremes for the period 2050 to 2100 than 2022 to 2050. An  investigation of the link between local knowledge and academic models showed that residency was the most important factor affecting participants' awareness of sea-level extremes.  By combining social resilience with natural and technological resilience it allows for a more a holistic understanding of changing resilience to SLEs within Trondheim.
   

\newpage

\section{Sammensdrag}
Forente nasjoners (FNs) mål 11 og 13 for bærekraftig utvikling stiller krav om økt resiliens til samfunn og byer. Som en kystnasjon har det stor betydning at Norge har en stor grad av resiliens imot ekstreme havnivåendringer. Tidligere studier har fokusert på nasjonal eller global risiko, men ekstreme havnivåendringer har i Norge en helt spesifikk lokalisering. Forståelsen av risiko må derfor være regionalt og lokalt forankret. Formålet med dette prosjektet er å bestemme grad av resiliens mot havnivåekstremer i fire viktige steder i Trondheim. Videre hjelpe til med å lage et rammeverk for raskt og billig å kunne bestemme et steds resiliens mot skiftende havnivåendringer. Beregnet resiliens betraktes som det estimerte resultatet av samspillet mellom naturlige, teknologiske og sosiale systemer. For grad av resiliens er de sosiale systemene den mest dynamiske faktoren, og fokuset i den masteroppgaven. Undersøkelser viser større bevissthet om risikoen for ekstreme havnivåer for perioden 2050 til 2100 enn 2022 til 2050. Vi fant at bosted var den viktigste faktoren som påvirket folks bevissthet om ekstreme havnivåer. Ved å kombinere sosial resiliens med naturlig og teknologisk resiliens kan du få et helhetlig syn på havnivåekstremer og motstandsendringer i Trondheim.