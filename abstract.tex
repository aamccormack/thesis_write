
%Abstract (one page, c.350 words)
%Short summary of the thesis
%• what you investigated
% how you did it
%what you found out

\addcontentsline{toc}{section}{Abstract}
\section{Abstract}

Improving the resilience of human settlements is required by United Nations Sustainable Development Goals (UN SDG's) 11 and 13. As a coastal country, the resilience to sea level extremes is of great importance to Norway. Previous studies have focused on national or global level risk, however sea level extremes are highly localized  and the understanding of risk needs to be at a regional and local levels. The purpose of this thesis is to determine resilience to sea level extremes in 4 key sites in Trondheim and contribute to the creation of a framework for quickly and inexpensively determining a place's resilience to changing sea level extremes. Projected resilience is determined from the interplay of natural, technological and social systems, which can all vary in space and time. By working along the boundary of physical and human geography, there is a hope of creating a more holistic projection of Trondheim's resilience. The most dynamic factors are those of the social systems of resilience and as such are the focus in this thesis. Stakeholder surveys conducted in Trondheim coastal neighbourhoods indicate better awareness of the risk of sea level extremes for the period 2050 to 2100 than 2022 to 2050. An  investigation of the link between local knowledge and academic models showed that residency was the most important factor affecting participants' awareness of sea-level extremes.  


**Here I think you need to add a summary statement about the combination of social resilience with natural (and maybe also technological resilience) in Trondheim since you state above that, in this thesis you are working along the boundary of physical and human geography - but then you only summarize  your human geography findings in the last two sentences. I know that this was your main focus in your study but then you need to either remove the statement about working along the boundary of human and physical geography (you can write about this idea in your Introduction or Background) or add something about the physical resilience. Your abstract should really be a short summary of your thesis so really shouldn't promise things that it cannot deliver.**

\newpage

\section{Sammensdrag}
Forbedring av motstandskraften av menneskelige bosetninger og byutvikling er krav fra de forente nasjoners (FNs) mål for bærekraftig utvikling 11 og 13. Som kystland er motstandskraften mot ekstreme havnivåer av stor betydning for Norge. Tidligere studier har fokusert på nasjonal eller global risiko, men ekstreme havnivåer er ekstremt lokaliserte i Norge. Forståelsen av risiko må være på regionalt og lokalt nivåer. Formålet med dette prosjektet er å bestemme motstandskraft mot havnivåekstremer i fire viktige steder i Trondheim og hjelpe til med å lage et rammeverk for raskt og billig å bestemme et steds motstandskraft mot skiftende havnivåekstremer. Prosjektert motstandskraft betraktes som det projiserte resultatet av samspillet mellom naturlige, teknologiske og sosiale systemer. Ved å inkludere fysisk og menneskelig geografi kan vi skape en mer avrundet idé om Trondheims motstandskraft.
Sosiale systemer for motstandskraft er den mest dynamiske faktoren og er det fokus i den masters oppgaven. Overraskende, undersøkelser resultater er subjekter har bedre bevissthet om risikoen for ekstreme havnivåer for perioden 2050 til 2100 enn 2022 til 2050. Under undersøkelsen av sammenhengen mellom lokalkunnskap og akademiske modeller var residens den viktigste faktoren som påvirket subjekts bevissthet om havnivåekstremer.

\section{Sammensdrag REWRITE}
Forente nasjoners (FNs) mål 11 og 13 for bærekraftig utvikling stiller krav om økt resiliens/ robusthet til samfunn og byer. Som en kystnasjon har det stor betydning at Norge har en stor grad av resiliens/robusthet imot ekstreme havnivåendringer. Tidligere studier har fokusert på nasjonal eller global risiko, men ekstreme havnivåendringer har i Norge en helt spesifikk lokalisering. Forståelsen av risiko må derfor være regionalt og lokalt forankret. Formålet med dette prosjektet er å bestemme grad av resiliens/ robusthet mot havnivåekstremer i fire viktige steder i Trondheim. Videre hjelpe til med å lage et rammeverk for raskt og billig å kunne bestemme et steds resiliens/ robusthet mot skiftende havnivåendringer. Beregnet resiliens/ robusthet betraktes som det estimerte resultatet av samspillet mellom naturlige, teknologiske og sosiale systemer. Ved å inkludere fysisk og menneskelig geografi kan vi skape en mer helhetlig idé om Trondheims grad av resiliens/ robusthet. 

For grad av resiliens/ robusthet er de sosiale systemene den mest dynamiske faktoren, og fokuset i den masteroppgaven. Overraskende, undersøkelser viser større bevissthet om risikoen for ekstreme havnivåer for perioden 2050 til 2100 enn 2022 til 2050. For sammenhengen mellom lokalkunnskap og teoretiske modeller var bosted den viktigste faktoren som påvirket folks bevissthet om havnivåekstremer.