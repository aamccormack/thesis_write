
%Abstract (one page, c.350 words)
%Short summary of the thesis
%• what you investigated
% how you did it
%what you found out

\addcontentsline{toc}{section}{Abstract}
\section{Abstract}

Improving the resilience of human settlements is required by United Nations Sustainable Development Goals (UN SDG's) 11 and 13. As a coastal country, the resilience to sea level extremes is of high importance to Norway. Previous studies have focused on national or global level risk, however sea level extremes are highly localized in Norway and the understanding of risk needs to be at a regional and local level. The purpose of this project is to determine resilience to sea level extremes in 4 key sites in Trondheim and assist the creation of a framework for quickly and cheaply determining a place's resilience to changing sea level extremes. Projected resilience is considered as the projected outcome of the interplay of natural, technological and social systems. By working in the boundary of physical and human geography, there is a hope of creating a more rounded view of Trondheim's resilience. The most dynamic factors are those of the social systems of resilience and as such are focused upon here. Surprisingly, surveys conducted indicate better awareness of the risk of sea level extremes for the period 2050 to 2100 than 2022 to 2050. During the investigation of the link between local knowledge and academic models, residency was the most important factor affecting participants' awareness of sea level extremes.   

\newpage

\section{Sammensdrag}
Forbedring av motstandskraften av menneskelige bosetninger og byutvikling er krav fra de forente nasjoners (FNs) mål for bærekraftig utvikling 11 og 13. Som kystland er motstandskraften mot ekstreme havnivåer av stor betydning for Norge. Tidligere studier har fokusert på nasjonal eller global risiko, men ekstreme havnivåer er ekstremt lokaliserte i Norge. Forståelsen av risiko må være på regionalt og lokalt nivåer. Formålet med dette prosjektet er å bestemme motstandskraft mot havnivåekstremer i fire viktige steder i Trondheim og hjelpe til med å lage et rammeverk for raskt og billig å bestemme et steds motstandskraft mot skiftende havnivåekstremer. Prosjektert motstandskraft betraktes som det projiserte resultatet av samspillet mellom naturlige, teknologiske og sosiale systemer. Ved å inkludere fysisk og menneskelig geografi kan vi skape en mer avrundet idé om Trondheims motstandskraft.
Sosiale systemer for motstandskraft er den mest dynamiske faktoren og er det fokus i den masters oppgaven. Overraskende, undersøkelser resultater er subjekter har bedre bevissthet om risikoen for ekstreme havnivåer for perioden 2050 til 2100 enn 2022 til 2050. Under undersøkelsen av sammenhengen mellom lokalkunnskap og akademiske modeller var residens den viktigste faktoren som påvirket subjekts bevissthet om havnivåekstremer.