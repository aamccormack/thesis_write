%This is the last chapter 
%%=========================================

%Abstract (one page, c.350 words)
%Short summary of the thesis
%• what you investigated
% how you did it
%what you found out

\addcontentsline{toc}{section}{Abstract}
\section{Abstract}

Norway’s resilience to sea level extremes is and has been a dynamic process. To better protect coastal infrastructure and populations an understanding of how this resilience is changing is required. Improving the resilience of human settlements is required by UN SDG's 11 and 13. Resilience is impacted by many factors which can be considered as falling within three key systems:  natural, technological and social systems. These systems interplay to alter the level of resilience and risk within places. Sea level extremes are extremely localised hazard in Norway and the understanding of this risk needs to be at a regional rather than national or global level. Changing climate, population distribution, geological setting change the resilience to sea level extremes that places in Norway have. 
The purpose of this project is to determine resilience to sea level extremes in 4 key places in Trondheim and assist the creation of a framework for quickly and cheaply determining a places resilience to changing sea level extremes. Investigating the link between local knowledge and academic models of sea level change in Trondheim is part of this process, including investigating which factors make individuals more aware of the risks of sea level extremes. This was done using a survey. *add sentance summarising results*

\newpage
\section{Sammensdrag}


