%This is the last chapter 
%%=========================================

%Abstract (one page, c.350 words)
%Short summary of the thesis
%• what you investigated
% how you did it
%what you found out

\addcontentsline{toc}{section}{Abstract}
\section{Abstract}

Improving the resilience of human settlements is required by UN SDG's 11 and 13. As a coastal country the resilience to sea level extremes is of high importance to Norway. Previous studies have focused on national or global level risk, however sea level extremes are extremely localized in Norway and the understanding of risk needs to be at a regional and local level. The purpose of this project is to determine resilience to sea level extremes in 4 key sites in Trondheim and assist the creation of a framework for quickly and cheaply determining a places resilience to changing sea level extremes. Projected resilience is considered as the projected outcome of the interplay of natural, technological and social systems. By working in the boundary of physical and human geography there is a hope of creating a more rounded view of Trondheim's resilience. Social systems of resilience are in many ways the most dynamic factor and as such are focused upon during this thesis. Surprisingly surveys conducted indicate better awareness of the risk of sea level extremes for the period 2050 to 2100 than 2022 to 2050. During the investigation of the link between local knowledge and academic models, residency was the most important factor affecting subjects awareness of sea level extremes.   



\section{Sammensdrag}


