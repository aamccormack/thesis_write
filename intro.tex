%%INTRODUCTION=========================================

%-UN SDGS DEMAND MORE RESILIENCE
%-RESILEINCE IS DYNAMIC AND CHANGING
%REQUIRE FRAMEWORK TO QUICKLY & CHEAPLY DETERMINE RESILIENCE
%-SO WE CAN SEE HOW IT CHANGES
%- THIS IS AN ATTEMPT TO ASSIST THE CREATION OF SUCH A FRAMEWORK 
%- ASPECTS OF RESILIENCE - NATURAL, SOCIAL, TECHNOLOGICAL SYSTEMS

%%=========================================
%Introduce the topic or phenomenon you want to study: why is it important to study this? State of the art

%Why is this relevant both for society and research?

%Previous research on this area and why your thesis brings new perspectives or knowledge (Here or in the theory section)

%In this section, you should present the problem that you are going to investigate or analyze; why this problem is of interest; what has, so far, been done to solve the problem, and which parts of the problem that remain.
%%=========================================
\chapter{Introduction}

\section{Project Purpose}
The overall purpose of this project is to determine the resilience to sea level extremes in 4 key places in Trondheim. However resilience is dynamic and to truly understand its changes would require a framework of periodic analysis. This project was created to assist the creation of such a framework, which could quickly and cheaply determine a places resilience to changing sea level extremes. Determining an overall understanding of an places resilience requires connecting local knowledge with academic knowledge. Additionally the method for determining resilience is designed to increase the awareness of the potential of sea level extremes in Trondheim. 

\section{Project Motivation}
 The focus of my project is on the boundary of physical and human geography and attempting to minimise the gap between these views on the changing sea level in Norway. The motivation behind this project is the realisation of the importance of local knowledge in increasing resilience. This is combined with meeting individuals in areas of uplift in the Nordics who have memory of how the sea levels have changed and the changing patterns of extreme sea levels in their places. This knowledge has been created from long term observation of places and appeared underutilised by academia and planners alike, but could be an helpful in the understanding of how areas will be impacted by the changing climate. 

\section{Project Objectives}
The project objective were to determine Trondheim's resilience to sea level extremes by considering social, natural and technological systems. To do this a survey was created to allow for determination of social systems of resilience. To determine natural systems resilience models of changing sea level extremes from kvartverket plus investigation of the coastline were used. To determine technological resilience planning permission from the municipality for each of the areas was analysed plus a consideration of how the infrastructure of the coastline impacts resilience.

\section{Research Question}
\begin{enumerate}
    \item How resilient will Trondheim be to SLE’s during 2022 to 2050 and 2050 to 2100?
    \item Are stakeholders aware about changes to SLE’s?
    \item What factors impact stakeholders’ awareness of SLEs ?
\end{enumerate}

\section{Hypothesis}
\begin{enumerate}
    \item We expect Trondheim to have high resilience to SLE's during 2022 to 2050
    \item We expect Trondheim to have medium resilience to SLE's during 2050 to 2100
    \item We expect stakeholders to be aware about changes to SLE'S.
\end{enumerate}
\paragraph{}
Hypothesis of which factors will impact stakeholders awareness of SLE'S.
Awareness is determined by the ability to correctly answer five questions about the changing sea levels in a specific place.

Hypothesis dependent on community membership
\begin{enumerate}
    \item We expect residents to be aware
    \item We expect commuters to be aware
    \item We expect marine workers to be aware
    \item We expect non-marine workers to be unaware
    \item We expect water leisure users to be aware
    \item We expect land leisure users to be unaware

    \end{enumerate}
\paragraph{}

Hypothesis dependent on local knowledge
\begin{enumerate}
    \item We expect subjects with professional interest in SLE's to be aware
    \item We expect subjects with primary knowledge about places which are on reclaimed land to be aware
    \item We expect subjects with a length of knowledge greater than 20 years of the area to be aware
    \item We expect subjects with a length of knowledge less than 1 year of the area to be aware
    \item We expect subjects with many information sources about the place to be more aware
    \item We expect subjects who chose to respond in Norwegian to be aware
\end{enumerate}

Hypothesis dependent on awareness of changing climate
\begin{enumerate}
    \item We expect subjects with many information sources about climate change to be more aware
    \item We expect subjects with the information source of formal education and/or reviewed published papers to be aware
    \item We expect subjects who are more concerned about climate change to be aware
    \item We expect subjects who predict they will be impacted by flooding from SLE's to be more aware
\end{enumerate}
%%=========================================

\section{Limitations}
In this section you describe the limitations of your study. These may be related to the study object (physical limitations, operational limitations), to the thoroughness of the analysis, and so on.

%seaonsally limited
%subjects limited - no marine workers
%
%%=========================================
\section{Approach}
Here you should describe the (scientific) approach that you will use to solve the problem and meet your objectives. You should specify the approach for each objective.

If there are any ethical problems related to your approach, these should be highlighted and discussed.
%%=========================================
\section{Structure of the Report}
The rest of the report is structured as follows. Chapter 2 gives an introduction to \ldots
%Readers guide
