%This is chapter 1
%%=========================================
\chapter{Introduction}
The first chapter of a well-structured thesis is always an introduction, setting the scene with background, problem description, objectives, limitations, and then looking ahead to summarize what is in the rest of the report. This is the part that readers look at first---\emph{so make sure it hooks them!}

%%=========================================
\section{Background}
In this section, you should present the problem that you are going to investigate or analyze; why this problem is of interest; what has, so far, been done to solve the problem, and which parts of the problem that remain.
%%=========================================
\subsection*{Problem Formulation}
You should define your problem in a clear an unambiguous way and explain why this is a problem, why it is of interest---and to whom. It is also important to delimit the problem area.
%%=========================================
\subsection*{Literature Survey}
You should here present the main books and articles that treat problems that are similar to what  you are studying. If you,  later in your thesis, describe the ``state of the art'' -- with a detailed literature survey, you may just give a very brief survey here (approx. a quarter of a page). If this is the only literature survey, you need to go into more details. An objective of the literature survey is to show the reader that you are familiar with the main literature within your field of research -- so that you do not ``reinvent the wheel.''


References to literature can be given in two different ways:
\begin{itemize}
\item As an \emph{explicit} reference: It is shown by \citet{lundteigen08} and partly also by \citet{rausand04}  that \ldots.
\item As an \emph{implicit} reference: It is shown \citep[e.g., see][Chap. 4]{rausand04} that \ldots.
\end{itemize}
In the example above, we have used ``author-year'' references, which is the preferred format. 
\begin{remark}
Following agreement with your supervisor, you may also refer by numbers, for example,  [1]. To do this, open the file \texttt{ramsstyle.sty} and  comment out (by \%) the command \texttt{$\backslash$usepackage\{natbib\}} and un-comment the corresponding command \texttt{$\backslash$usepackage[numbers]\{natbib\}}.\footnote{Notice the strange way we have to write the ``backslash'' in the text. This is because the ``backslash'' is a command in \LaTeX.}
\end{remark}
 You may include a link to the Internet in the text or in a footnote by using a command like: \url{http://www.ntnu.edu/ross}. 

When you refer to the scientific literature, you should always write in \emph{present} tense. Example: \citet{rausand04} show that \ldots.

\begin{remark}
Hyperlinks are included by the command \texttt{$\backslash$usepackage\{hyperref}\} in \texttt{ramsstyle.sty}. If you feel that the hyperlinks are disturbing when you enter the text, or want to avoid the hyperlinks in printed text, you may either comment out or edit this command in \texttt{ramsstyle.sty}.
\end{remark}
%%=========================================
\subsection*{What Remains to be Done?}
After you have defined and delimited your problem -- and presented the relevant results found in the literature within this field, you should sum up which parts of the problem that remain to be solved.
%%=========================================
\section{Objectives}
The main objectives of this Master's project are
\begin{enumerate}
\item This is the first objective
\item This is the second objective
\item This is the third objective
\item More objectives
\end{enumerate}

All objectives shall be stated such that we, after having read the thesis, can see whether or not you have met the objective. ``To become familiar with \ldots'' is therefore not a suitable objective.

%%=========================================
\section{Limitations}
In this section you describe the limitations of your study. These may be related to the study object (physical limitations, operational limitations), to the thoroughness of the analysis, and so on.
%%=========================================
\section{Approach}
Here you should describe the (scientific) approach that you will use to solve the problem and meet your objectives. You should specify the approach for each objective.

If there are any ethical problems related to your approach, these should be highlighted and discussed.
%%=========================================
\section{Structure of the Report}
The rest of the report is structured as follows. Chapter 2 gives an introduction to \ldots

\begin{remark}
Notice that chapter and section headings shall be written in lowercase, but that all main words should start with a capital letter.
\end{remark}


The report should be no longer than \underline{60 pages} in this format (+ the CV).