%%INTRODUCTION=========================================

%-UN SDGS DEMAND MORE RESILIENCE
%-RESILEINCE IS DYNAMIC AND CHANGING
%REQUIRE FRAMEWORK TO QUICKLY & CHEAPLY DETERMINE RESILIENCE
%-SO WE CAN SEE HOW IT CHANGES
%- THIS IS AN ATTEMPT TO ASSIST THE CREATION OF SUCH A FRAMEWORK 
%- ASPECTS OF RESILIENCE - NATURAL, SOCIAL, TECHNOLOGICAL SYSTEMS

%%=========================================
\section{Background}

%Introduce the topic or phenomenon you want to study: why is it important to study this? State of the art

%Why is this relevant both for society and research?

%Previous research on this area and why your thesis brings new perspectives or knowledge (Here or in the theory section)

%In this section, you should present the problem that you are going to investigate or analyze; why this problem is of interest; what has, so far, been done to solve the problem, and which parts of the problem that remain.
%%=========================================

\section{Project Purpose}
To determine the resilience to sea level extremes in 4 key places in Trondheim. To assist creation of framework for quickly and cheaply determining a places resilience to changing sea level extremes. to connect local knowledge to academic knowledge and assist in increasing awareness of the potential of sea level extremes in Trondheim. 

\section{Project Objectives}

\section{Research Question}
%\boldsymbol{R1} How resilient will Trondheim be to SLE’s during 2022 to 2050 and 2050 to 100
%\boldsymbol{R2}Are stakeholders aware about changes to SLE’s?
%\boldsymbol{R3}What factors impact stakeholders’ awareness of SLEs ?


BREAKS DOWN INTO HYPOTHESIS
%%=========================================

\section{Limitations}
In this section you describe the limitations of your study. These may be related to the study object (physical limitations, operational limitations), to the thoroughness of the analysis, and so on.

%seaonsally limited
%subjects limited - no marine workers
%
%%=========================================
\section{Approach}
Here you should describe the (scientific) approach that you will use to solve the problem and meet your objectives. You should specify the approach for each objective.

If there are any ethical problems related to your approach, these should be highlighted and discussed.
%%=========================================
\section{Structure of the Report}
The rest of the report is structured as follows. Chapter 2 gives an introduction to \ldots
%Readers guide
