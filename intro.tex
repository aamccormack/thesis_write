%%INTRODUCTION=========================================

%-UN SDGS DEMAND MORE RESILIENCE
%-RESILEINCE IS DYNAMIC AND CHANGING
%REQUIRE FRAMEWORK TO QUICKLY & CHEAPLY DETERMINE RESILIENCE
%-SO WE CAN SEE HOW IT CHANGES
%- THIS IS AN ATTEMPT TO ASSIST THE CREATION OF SUCH A FRAMEWORK 
%- ASPECTS OF RESILIENCE - NATURAL, SOCIAL, TECHNOLOGICAL SYSTEMS

%%=========================================
%Introduce the topic or phenomenon you want to study: why is it important to study this? State of the art

%Why is this relevant both for society and research?

%Previous research on this area and why your thesis brings new perspectives or knowledge (Here or in the theory section)

%In this section, you should present the problem that you are going to investigate or analyze; why this problem is of interest; what has, so far, been done to solve the problem, and which parts of the problem that remain.
%%=========================================
\chapter{Introduction}
\section{Introduction to Project}
Coastal flooding due to extreme weather and sea level rise is increasing, because of the changing climate (\cite{ipcc_sea_2021}).  In other words, Sea Level Extremes (SLEs) are becoming more frequent, globally. The majority of Norwegian critical infrastructure and population is located within the coastal zone, meaning it is within 100m of the high tide (\cite{engebakken_construction_2022}), and thus, more frequent SLEs may have significant impacts in Norway. However, compared to global standards of coastal countries, Norway appears well-protected from the global sea-level rise caused by climate change (\cite{aunan_strong_2008}). While the increase in area permanently inundated with water will be minimal within the next 70 years, there is likely to be an increase in extreme weather which will cause temporary SLEs (\cite{aunan_strong_2008}). The localised impacts of natural hazards which are increasing due to the changing climate such as SLE events, requires greater focus to allow for protection of people and assets (\cite{lujala_quantifying_2014} ;\cite{aunan_strong_2008}). Previous models of changing SLEs were often conducted at too low a resolution, or were too simplistic to fully understand potential impacts to critical infrastructure, such as hospitals or major transport routes (\cite{hoffken_effects_2020}). Due to newer models and more detailed mapping from Kartverket (\cite{kartverket_se_2021}), the understanding of the impacts from potential SLEs is much clearer for the period 2022 to 2100. Whether this new information is aligned with local understanding of SLEs (considered here under the broad term awareness as discussed in Theory Section \ref{theory-resilience}), impacts resilience to this risk (\cite{setten_we_2019}).  There is a need for localised understanding of this changing risk and how this changes the resilience of places (\cite{rod_integrated_2012}).
\paragraph{}

\section{Thesis Motivation}
 This thesis focuses on the interplay between local knowledge and more formalised knowledge such as scientific and institutional knowledge. The scientific knowledge is often either grounded in mathematical models of threats or theoretical underpinnings, rather than lived experienced of stakeholders (\cite{gerkensmeier_governing_2018}). Furthermore, this knowledge has difficulty being utilised by policymakers (\cite{gerkensmeier_governing_2018}). The motivation behind this thesis is the personal realisation of the importance of local knowledge in increasing resilience gained from the courses taken as part of the Natural Resources Management MSc and highlighted by international frameworks including the United Nations Sustainable Development Goals (UN SDGs) (\cite{un_sustainable_2021}).  

\section{Thesis Purpose}
The overall purpose of this thesis is to determine the resilience to SLEs in four coastal neighbourhoods in Trondheim. However, resilience is dynamic and to truly understand it would require a framework of periodic analysis. This thesis was developed to assist in the creation of such a framework, which could quickly and inexpensively determine a place's resilience to changing SLEs. Determining an overall understanding of a place's resilience requires connecting local knowledge with academic knowledge.  


\section{Thesis Objectives}
The thesis objective is to determine Trondheim's resilience to SLEs by considering social, natural and technological systems. To do this, a survey was created to allow for determination of social systems of resilience.  Models of changing SLEs were used (\cite{kartverket_se_2021}), plus investigation of the coastline, to determine natural systems resilience. To determine technological systems resilience, city plans, building regulations and planning permissions from the municipality for each of the research sites were reviewed and the impacts of coastal infrastructure on resilience were analysed.

\section{Research Questions}
\begin{enumerate}
    \item How resilient will Trondheim be to SLEs during the period 2022 to 2050 and 2050 to 2100?
    \item Are Trondheim's coastal stakeholders aware of changes to SLEs?
    \item What factors impact  Trondheim's coastal stakeholders’ awareness of SLEs?
\end{enumerate}


\section{Hypotheses for Research Questions}\label{hypotheses-intro}

\subsection{Research Question 1}
It was hypothesised that Trondheim is highly resilient to SLEs during the period of 2022 to 2050. \cite{opach_seeking_2020} determined that Trondheim has high community resilience due to its social, economic and infrastructural resources, this is combined with it location (figure \ref{fig:research_area}) which protects it from the majority of storms coming off the North sea. It was hypothesised that due to the changing climate (\cite{ipcc_sea_2021}), decreasing uplift rates (\cite{hanssen-bauer_climate_2017}) and social shifts (\cite{dsb_integrating-sea-level-rise-and-storm-surges--local-planningpdf_2017}) including increasing population in Trondheim (\cite{ssb_kommunefakta_2023}) that Trondheim will then decrease to quite resilient during the period of 2050 to 2100. 
\paragraph{}

 
\subsection{Research Question 2}
It was hypothesised that Trondheim's coastal stakeholders are very aware of changes to SLEs, due to the assumed ability to directly observe these changes, the high levels of employment in the marine sector and the high levels of educated individuals in the city (\cite{opach_seeking_2020}; \cite{lujala_role_2020}).

\subsection{Research Question 3}
Factors  impacting stakeholder awareness of SLEs were tested using non-parametric hypothesis testing with the Kruskal Wallis Rank Sum Test (\cite{hollander_nonparametric_2014}), where awareness is determined by the ability to answer up to five questions about the changing sea levels in a specific place and identify which answers correspond with the models from \cite{kartverket_se_2021}. Whether these models projecting future SLEs match with reality remains to be seen, which is why the discussion will be phrased in terms of corresponding with the model rather than answering correctly. If the answers do not match the models this highlights a need for future research to find out why this is not the case. If stakeholders local knowledge is closer to reality of changing SLEs than the models this opens up an interesting opportunity for improving these models. If the stakeholders lack awareness about SLEs then there is an opportunity for improving social resilience to SLEs via education and exposure to these models in a comprehend-able way. Outlined below is the hypotheses of which factors awareness will be dependent upon. These were created prior to testing. The outcomes are detailed at the end of the results section. 

\begin{enumerate}
    \item It is hypothesised that stakeholder awareness of SLEs will vary depending on community membership: we expect residents, commuters, marine workers and water leisure users to have higher determined awareness than non-marine workers and land leisure users.
  
    \item It is hypothesised that stakeholder awareness of SLEs will vary depending on local knowledge: we expect subjects with professional interest in SLEs,  primary knowledge about places on reclaimed land, length of knowledge of over 20 years, many information sources about the place and those who respond in Norwegian to have a higher determined awareness. We expect subjects who are new to the area (less than 1 year length of knowledge) to have a lower determined awareness.

    \item It is hypothesised that stakeholder awareness of SLEs will vary depending on their awareness of changing climate: we expect subjects who have access to information sources from formal education, for example, peer-reviewed publications, to have higher determined awareness. Additionally, we expect those who are more concerned about climate change or who predict they will be impacted by flooding from SLEs to have higher determined awareness. 
\end{enumerate}

%%=========================================
\section{Approach}

This thesis has been developed using a mixed methodological approach of human and physical geography. The integration of these is essential to understanding resilience to a natural hazard. Concerns of labeling an area as not resilient or not aware were considered carefully before the start of this project.  It was deemed important that improving resilience and awareness needed to be the end goal. For this reason, this research creates a framework for measuring resilience. The ethics of how to improve resilience must also be considered. It was an active choice not to investigate demographics, such as gender, immigration status, age and physical ability, due to the ethical and political implications of making policy based on such results. While the highlighting of vulnerable populations and factors is important and often only identified after disasters (\cite{cutter_community_2020}), to declare these factors as undesirable based solely on their projected resilience to specific hazards is problematic. 

\paragraph{}
Factors affecting resilience which could be impacted by municipal planning or communications were prioritised. There is a need for a framework which can quickly and inexpensively provide an overview of the resilience for places in Norway, which can then be used to create adaptive strategies by policy makers (\cite{opach_seeking_2020}). There have been attempts to create vulnerability indices, but many of these include warnings that they should not be used to inform policy (\cite{opach_seeking_2020}). 


%%=========================================
\section{Thesis Structure}
The following thesis is organised into eight chapters. The purpose of this chapter was to introduce the purpose of the thesis and the research questions.

In chapter two the background of the thesis is outlined including the research area and the specific sites. 

Chapter three is the theory section which begins with defining the key terms and then goes on to explain the theories of resilience used during this project.

Chapter four outlines the methodology used including the data collection, data analysis and the limitations of the methodology.

Chapter five is the results of the data collected and the results of the tests ran on this data, including the results of Kruskal Wallis Rank Sum Test. 

Chapter six is a discussion of the results presented in chapter five and included discussion of the research questions presented in the introduction. 

Chapter seven is a discussion of the Framework for Determining Resilience. The purpose of this thesis was to assist the creation of such a framework and this section discusses the lessons learned including about conducting citizen science. It also includes suggestions for future research.

Finally chapter eight is the Conclusion section which summarises the results from each of the sections outlined here.

Chapter eight is followed by the appendix which is split into three sections. Section A includes the acronyms used throughout the thesis. Section B includes additional information which focuses on the visual aspects of communication, which were a key aspect of the design method for this project. Section C is a copy of the survey used in both English and Norwegian to for allow easy repetition of this project. 
