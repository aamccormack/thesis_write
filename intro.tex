%%INTRODUCTION=========================================

%-UN SDGS DEMAND MORE RESILIENCE
%-RESILEINCE IS DYNAMIC AND CHANGING
%REQUIRE FRAMEWORK TO QUICKLY & CHEAPLY DETERMINE RESILIENCE
%-SO WE CAN SEE HOW IT CHANGES
%- THIS IS AN ATTEMPT TO ASSIST THE CREATION OF SUCH A FRAMEWORK 
%- ASPECTS OF RESILIENCE - NATURAL, SOCIAL, TECHNOLOGICAL SYSTEMS

%%=========================================
%Introduce the topic or phenomenon you want to study: why is it important to study this? State of the art

%Why is this relevant both for society and research?

%Previous research on this area and why your thesis brings new perspectives or knowledge (Here or in the theory section)

%In this section, you should present the problem that you are going to investigate or analyze; why this problem is of interest; what has, so far, been done to solve the problem, and which parts of the problem that remain.
%%=========================================
\chapter{Introduction}
\section{Introduction to Project}
Coastal flooding due to extreme weather and sea level rise is increasing due to the changing climate \cite{hoffken_effects_2020}.  In other words sea level extremes are becoming more frequent globally. The majority of Norwegian critical infrastructure and population falls within the coastal zone \cite{engebakken_construction_2022}, meaning this change could have significant impacts. There is a narrative that Norway is protected from the global sea level rise caused by climate change, but while the amount of area permanently inundated with water is minimal in the next 70 years there is likely to be an increase in extreme weather which will cause temporary sea level extremes \cite{aunan_strong_2008}. The localised impact of these sea level extremes requires greater focus. Previous models of changing sea level extremes were often done at too large a scale, or too simplistic to understand these changes \cite{hoffken_effects_2020}. Due to newer models and better mapping the picture of potential sea level extremes is much clearer between now and 2100. Whether this new information is aligned with local understanding of sea level extremes impacts resilience to this risk\cite{setten_we_2019}.  There is a need for localized understanding of this changing risk and how this changes the resilience of places \cite{rod_integrated_2012}.
\paragraph{}

\section{Project Purpose}
The overall purpose of this project is to determine the resilience to sea level extremes in 4 key places in Trondheim. However resilience is dynamic and to truly understand its changes would require a framework of periodic analysis. This project was created to assist the creation of such a framework, which could quickly and cheaply determine a places resilience to changing sea level extremes. Determining an overall understanding of an places resilience requires connecting local knowledge with academic knowledge. Additionally the method for determining resilience is designed to increase the awareness of the potential of sea level extremes in Trondheim. 

\section{Project Motivation}
 The focus of my project is on the boundary of physical and human geography and attempting to minimise the gap between these views on the changing sea level in Norway. The motivation behind this project is the realisation of the importance of local knowledge in increasing resilience. This is combined with meeting individuals in areas of uplift in the Nordics who have memory of how the sea levels have changed and the changing patterns of extreme sea levels in their places. This knowledge has been created from long term observation of places and appeared underutilised by academia and planners alike, but could be an helpful in the understanding of how areas will be impacted by the changing climate. 

\section{Project Objectives}
The project objective were to determine Trondheim's resilience to sea level extremes by considering social, natural and technological systems. To do this a survey was created to allow for determination of social systems of resilience. To determine natural systems resilience models of changing sea level extremes from kvartverket plus investigation of the coastline were used. To determine technological resilience planning permission from the municipality for each of the areas was analysed plus a consideration of how the infrastructure of the coastline impacts resilience.

\section{Research Question}
\begin{enumerate}
    \item How resilient will Trondheim be to SLE’s during 2022 to 2050 and 2050 to 2100?
    \item Are stakeholders aware about changes to SLE’s?
    \item What factors impact stakeholders’ awareness of SLEs ?
\end{enumerate}



\section{Hypothesis}
Hypothesis testing  of which factors will impact stakeholders awareness of SLE'S was conducted using a non parametric hypothesis testing using the Kruskal Wallis Rank Sum Test. For this Awareness is determined by the ability to correctly answer five questions about the changing sea levels in a specific place. Outlined below is the hypothesis of which factors, awareness will be dependent upon prior to testing. 

\subsection{Hypothesis dependent on community membership}
\begin{enumerate}
    \item We expect residents to be aware
    \item We expect commuters to be aware
    \item We expect marine workers to be aware
    \item We expect non-marine workers to be unaware
    \item We expect water leisure users to be aware
    \item We expect land leisure users to be unaware
    \end{enumerate}
\paragraph{}

\subsection{Hypothesis dependent on local knowledge}
\begin{enumerate}
    \item We expect subjects with professional interest in SLE's to be aware
    \item We expect subjects with primary knowledge about places which are on reclaimed land to be aware
    \item We expect subjects with a length of knowledge greater than 20 years of the area to be aware
    \item We expect subjects with a length of knowledge less than 1 year of the area to be aware
    \item We expect subjects with many information sources about the place to be more aware
    \item We expect subjects who chose to respond in Norwegian to be aware
\end{enumerate}
\paragraph{}

\subsection{Hypothesis dependent on awareness of changing climate}
\begin{enumerate}
    \item We expect subjects with many information sources about climate change to be more aware
    \item We expect subjects with the information source of formal education and/or reviewed published papers to be aware
    \item We expect subjects who are more concerned about climate change to be aware
    \item We expect subjects who predict they will be impacted by flooding from SLE's to be more aware
\end{enumerate}
%%=========================================



%%=========================================
\section{Approach}

The approach to this project was to create an overview of resilience. This was done by actively including approaches used in physical geography and human geography. The overlap of these two views understanding is essential to understanding resilience to a natural hazard. The potential of this thesis resulting in labeling an area not resilient and not aware were considered carefully before the start of this project. When considering this it was deemed important that improving resilience and awareness needed to be the end goal. For this reason this research is framed as an attempt to assist the creation of a framework of measuring. Furthermore that the ethics of how to improve these factors needed to be considered. The conclusion of this was that the factors researched which may affect resilience were carefully considered. It was an active choice not to investigate demographics which it could be considered unethical to change in pursuit of resilience. For this reason questions on gender, immigration status, age and physical ability were excluded. Factors which could be impacted by municipal planning or communications were focused upon. 


%%=========================================
\section{Structure of the Report}
The rest of the report is structured as follows. 
The following thesis is organised into eight chapters. The purpose of this chapter was to introduce the purpose of the project and the research questions.

In chapter two the background of the project is outlined including the researh area and the specific sites. 

Chapter three outlines the methodology used including the data collection, data analysis and the limitations of the methodology.

Chapter four is the theory section which begins with defining the key terms and then goes on to explain the theories of resilience used during this project.

Chapter five is the results of the data collected and the results of the tests ran on this data. This is followed by chapter sx which is a discussion of the results presented in chapter five. 

Chapter seven is a discussion of the Framework for Determining Resilience. The purpose of this project was to assist the creation of such a framework and this section discusses the lessons learned including about conducting citizen science. It also includes suggestions for future research

Finally chapter eight is the Conclusion section which summarises the results from each of the sections outlined here.

This chapter is followed by the appendix which is split into three sections. Section A is the acronyms used within this thesis. Section be is additional information which focuses on the visual aspects of communication which were a key aspect of the design method for this project. Section C is a copy of the survey used in both English and Norwegian so when references are made to the survey design in the methodology and  discussion of framework a clearer idea of how subjects were presented with information and questions can be gained. This is also here to allow easy repetition of this project. This thesis can also be considered a guide to conducting this form of research in Trondheim. 
