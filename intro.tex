%%INTRODUCTION=========================================

%-UN SDGS DEMAND MORE RESILIENCE
%-RESILEINCE IS DYNAMIC AND CHANGING
%REQUIRE FRAMEWORK TO QUICKLY & CHEAPLY DETERMINE RESILIENCE
%-SO WE CAN SEE HOW IT CHANGES
%- THIS IS AN ATTEMPT TO ASSIST THE CREATION OF SUCH A FRAMEWORK 
%- ASPECTS OF RESILIENCE - NATURAL, SOCIAL, TECHNOLOGICAL SYSTEMS

%%=========================================
%Introduce the topic or phenomenon you want to study: why is it important to study this? State of the art

%Why is this relevant both for society and research?

%Previous research on this area and why your thesis brings new perspectives or knowledge (Here or in the theory section)

%In this section, you should present the problem that you are going to investigate or analyze; why this problem is of interest; what has, so far, been done to solve the problem, and which parts of the problem that remain.
%%=========================================
\chapter{Introduction}

\section{Project Purpose}
The overall purpose of this project is to determine the resilience to sea level extremes in 4 key places in Trondheim. However resilience is dynamic and to truly understand its changes would require a framework of periodic analysis. This project was created to assist the creation of such a framework, which could quickly and cheaply determine a places resilience to changing sea level extremes. Determining an overall understanding of an places resilience requires connecting local knowledge with academic knowledge. Additionally the method for determining resilience is designed to increase the awareness of the potential of sea level extremes in Trondheim. 

\section{Project Motivation}
The motivation behind this project is the realisation of the importance of local knowledge in increasing resilience. This is combined with meeting individuals in areas of uplift in the Nordics who have memory of how the sea levels have changed and the changing patterns of extreme sea levels in their places. This knowledge has been created from long term observation of places and appeared underutilised by academia and planners alike. 

\section{Project Objectives}
The project objective were to determine Trondheim's resilience to sea level extremes by considering social, natural and technological systems. 

\section{Research Question}
\begin{enumerate}
    \item How resilient will Trondheim be to SLE’s during 2022 to 2050 and 2050 to 100?
    \item Are stakeholders aware about changes to SLE’s?
    \item What factors impact stakeholders’ awareness of SLEs ?
\end{enumerate}

BREAKS DOWN INTO HYPOTHESIS
%%=========================================

\section{Limitations}
In this section you describe the limitations of your study. These may be related to the study object (physical limitations, operational limitations), to the thoroughness of the analysis, and so on.

%seaonsally limited
%subjects limited - no marine workers
%
%%=========================================
\section{Approach}
Here you should describe the (scientific) approach that you will use to solve the problem and meet your objectives. You should specify the approach for each objective.

If there are any ethical problems related to your approach, these should be highlighted and discussed.
%%=========================================
\section{Structure of the Report}
The rest of the report is structured as follows. Chapter 2 gives an introduction to \ldots
%Readers guide
